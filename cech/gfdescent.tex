\documentclass[11pt]{amsart}

\pdfoutput = 1

\usepackage{macros,slashed,amsaddr}

\linespread{1.5}

\setcounter{tocdepth}{2}
\numberwithin{equation}{section}
\newcommand{\nocontentsline}[3]{}
\newcommand{\tocless}[2]{\bgroup\let\addcontentsline=\nocontentsline#1{#2}\egroup}
\newcommand{\changelocaltocdepth}[1]{%
  \addtocontents{toc}{\protect\setcounter{tocdepth}{#1}}%
  \setcounter{tocdepth}{#1}%
}
\setcounter{tocdepth}{1}

\def\brian{\textcolor{blue}{BW: }\textcolor{blue}}
\def\ingmar#1{{\textcolor{green!65!black}{IAS: {#1}}}}

\def\Dol{{\rm Dol}}
\def\dR{{\rm dR}}
\def\ep{{\varepsilon}}
\def\define{\overset{\rm def}{=}}
\newcommand{\defterm}[1]{\textbf{\emph{#1}}}


\begin{document}

\title{Descent and Gelfand--Fuks cohomology}

The dg Lie algebra $\cT(\CC^n) = \Omega^{0,\bu}(\CC^n, T)$ is a resolution for the Lie algebra $\cT^{hol}(\CC^n)$ of holomorphic vector fields on $\CC^n$. 
There is a quasi-isomorphism
\[
p : \cT(\CC^n) \xto{\simeq}  H^0(\cT(\CC^n)) = \cT^{hol}(\CC^n) .
\]
induced by the projection onto the zeroeth cohomology of $\cT(\CC^n)$. 
Furthermore, there is a map 
\[
j_0^\infty :  \cT^{hol}(\CC^n) \to \fw_n
\]
which takes the Taylor expansion of a holomorphic vector field at $0 \in \CC^n$. 
The composition $j \define j_0^\infty \circ p : \cT(\CC^n) \to \fw_n$ is a map of dg Lie algebras. 

The map $j$ defines a map on (continuous) Chevalley--Eilenberg complexes
\[
j^* : \clie^\bu(\fw_n) \to \clie^\bu\left(\Omega^{0,\bu}(\cT(\CC^n))\right) .
\]

\begin{lem}
The map $j^*$ factors through $\clie^\bu\left(J \cT(\CC^n) \right) \hookrightarrow \clie^\bu(\cT(\CC^n))$.
\end{lem}

As a graded vector space $\clie^{\#}(J \cT(\CC^n))$ is the space of global sections of a graded (infinite rank) vector bundle on $\CC^n$ that we denote by $\clie^{\#}(J \cT)$. 
Equipped with the Chevalley--Eilenberg differential $\clie^\bu(J \cT)$ becomes a complex of vector bundles. 

By the lemma, we obtain for each $\phi \in \clie^\bu(\fw_n)$ a global section $j^* \phi$ of the vector bundle $\clie^\bu(J \cT)$. 

\begin{eg}
Suppose $n=1$ and consider the $1$-cochain $\phi : f \frac{\partial}{\partial z} \mapsto f'(0)$ of $\fw_1$. 
The value of the section $j^* \phi$ at the point $z_0 \in \CC$ is the cochain for $\cT = \Omega^{0,\bu}(\CC, T_\CC)$ defined by
\[
a(z,\zbar) \frac{\partial}{\partial z} + b(z,\zbar) \d \zbar \frac{\partial}{\partial z} \mapsto \frac{\partial}{\partial z} a(z,\zbar) |_{z = z_0} .
\]
\end{eg}

The sheaf of sections of the bundle $\clie^\bu(J \cT)$ is a sheaf of commutative dg algebras. 
In fact, it is a commutative dg algebra in the category of $D_{\CC^n}$-modules.

Consider the de Rham complex of the $D_{\CC^n}$-algebra $\clie^\bu(J \cT)$
\[
\Omega^\bu \left(\CC^n , \clie^\bu(J \cT) \right) .
\]

\begin{thm}
Suppose $\phi \in \clie^\bu(\fw_n)$ and let $\phi^0 = j^* \phi$. 
Then, there exists $\phi^{i,j} \in \Omega^{i,j} (\CC^n , \clie^\bu(J \cT))$, $1 \leq i,j \leq n$ such that the element 
\[
\Phi \define \sum_{i,j} \phi^{i,j} 
\]
satisfies the equation $(\d_{\rm dR} + \d_{\cT}) \Phi = 0$. 
\end{thm}

Using the Hodge decomposition of the Rham differential $\d_{\rm dR} = \dbar + \partial$, we we will actually show that the elements $\phi^{i,j}$ satisfy a pair of descent equations:
\begin{itemize}
\item Holomorphic descent:
\[
\dbar \phi^{i,j} = \dbar_{\cT} \phi^{i, j+1} 
\]
for $0 \leq i , j \leq n$ and
\item Cartan descent:
\[
\partial \phi^{i,j} = \d_{{\rm CE}} \phi^{i+1, j} 
\]
for $0 \leq i , j \leq n$. 
\end{itemize}

\begin{thm}
The assignment $\phi \mapsto \phi^{n,n}$ defines a quasi-isomorphism $\clie^\bu(\fw_n)[2n] \simeq \cloc^\bu(\cT)$.
In particular, if $\phi \in \clie^\bu(\fw_n)$ is a cocycle then $\phi^{n,n} \in \cloc^\bu(\cT)$ is a local cocycle and up to equivalence all such local cocycles are obtained in this way.
\end{thm}

\begin{eg}
Consider the Gelfand--Fuks cocycle $\phi \in \clie^3(\fw_n)$ defined by
\[
\phi \bigg(f(x) \frac{\d}{\d x} , g(x) \frac{\d}{\d x} , h(x) \frac{\d}{\d x} \bigg) = \brian{you know the determinant} .
\]
The section $\phi^0 = j^* \phi$ of $\clie^\bu(J \cT)$ is
\[
\phi^0 \bigg(\alpha(z,\zbar) \frac{\partial}{\partial z} , \beta(z,\zbar) \frac{\partial}{\partial z} , \gamma(z ,\zbar) \frac{\partial}{\partial z} \bigg) = 
\]
We first solve for the descent element $\phi^{0,1}$ which satisfies 
\[
\dbar \phi^0 = \d_{\cT} \phi^{0,1} = (\d_{\rm CE} + \dbar_\cT) \phi^{0,1} 
\]
which has the general formula
\[
\phi^{0,1} = \d \zbar \frac{\partial}{\partial (\d \zbar)} \phi^0 .
\]
Then, automatically $\d_{\rm CE} \phi^{0,1} = 0$.

\end{eg}

\appendix

\section{Smooth version}

Let's first consider the smooth case. 
Suppose $M$ is an $n$-dimensional manifold and let $\fX(M)$ denote the associated Lie algebra of vector fields. 
For each $x \in M$ we have a cochain map
\[
j_x^* : \clie^\bu(\fw_n) \to \clie^\bu(\fX(M))
\]
which sends a cochain $\alpha$ to $j_x^* \alpha$ where 
\[
j_x : \fX(M) \to \fw_n
\]
takes a vector field and computes its $\infty$-jet at $x \in M$. 

\begin{lem}
Let $\alpha \in \clie^k(\fw_n)$ be a cochain. 
For any $x,y \in M$ the cochains $j_x^* \alpha$ and $j_y^* \alpha$ are cohomologous. 
In particular, there exists a $\alpha^{(1)} \in \Omega^1(M) \otimes \clie^{k-1}(\fw_n)$ such that $(\d_{dR} \otimes 1) \alpha = (1 \otimes \d_{\fX}) \alpha^{(1)}$. 
\end{lem}

Inductively, we obtain a sequence of cochains $(j_x^* \alpha, \alpha^{(1)}, \ldots, \alpha^{(n)})$ satisfying 
\[
(\d_{dR} \otimes 1) \alpha^{(j)} = (1 \otimes \d_{\fX}) \alpha^{(j+1)} .
\]
It follows immediately that for any $\ell$-cycle $N \subset M$ one obtains a cochain of $\fX(M)$ with {\em trivial} coefficients:
\[
\int_N \alpha^{(\ell)} \in \clie^{k-\ell}(\fX(M)) .
\]

\begin{lem}
If $\alpha \in \clie^k(\fw_n)$ is a cocycle then $\int_N \alpha^{(\ell)}$ is a cocycle. 
\end{lem}

\begin{eg}
Consider the cocycle $\alpha \in \clie^3(\fw_1)$ which is dual to the homology $3$-cycle
\[
L_{-1} \wedge L_0 \wedge L_1 \in {\rm C}_3(\fw_1) .
\]
Consider $M = S^1$. 
For $x \in S^1$ one has 
\[
j_x^* \alpha \bigg(f(x) \frac{\d}{\d x} , g(x) \frac{\d}{\d x} , h(x) \frac{\d}{\d x} \bigg) = \brian{you know the determinant} .
\]
An explicit form of a cochain $\alpha^{(1)} \in \Omega^1(S^1) \otimes \clie^2(\fX(S^1))$ satisfying $\d_{\fX} \alpha^{(1)} = \d_{dR} j_x^* \alpha$ is
\[
\alpha^{(1)} \bigg( f(x) \frac{\d}{\d x} , g(x) \frac{\d}{\d x} \bigg) \define f'(x) \d_{dR} (g'(x)) - g'(x) \d_{dR} (f'(x)) \in \Omega^1(S^1) .
\]
As one can immediately check, $\int_{S^1} \alpha^{(1)} \in \clie^2 (\fX(S^1))$ is the usual Virasoro cocycle.
\end{eg}



\end{document}
