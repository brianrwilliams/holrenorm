\documentclass[11pt]{amsart}

\pdfoutput = 1

\usepackage{macros,slashed,amsaddr}

\linespread{1.5}

\setcounter{tocdepth}{2}
\numberwithin{equation}{section}
\newcommand{\nocontentsline}[3]{}
\newcommand{\tocless}[2]{\bgroup\let\addcontentsline=\nocontentsline#1{#2}\egroup}
\newcommand{\changelocaltocdepth}[1]{%
  \addtocontents{toc}{\protect\setcounter{tocdepth}{#1}}%
  \setcounter{tocdepth}{#1}%
}
\setcounter{tocdepth}{1}

\def\brian{\textcolor{blue}{BW: }\textcolor{blue}}
\def\ingmar#1{{\textcolor{green!65!black}{IAS: {#1}}}}

\def\Dol{{\rm Dol}}
\def\dR{{\rm dR}}
\def\Dol{{\rm Dol}}
\def\dR{{\rm dR}}
\def\vC{{\rm\check{C}}}
\def\vI{\check{I}}
\def\vomega{\check{\omega}}

\begin{document}

\title{Cech field theory}

\author{Brian R. Williams}
\address{Department of Mathematics, Northeastern University \\ 567 Lake Hall \\ Boston, MA 02115 \\ U.S.A.}
\email{br.williams@northeastern.edu}

\maketitle

\section{The general construction}

Let $E, F$ be two holomorphic vector bundles on $X$. 
Define the subspace
\[
\Omega^{(p,q);(r,s)} (X \times X , V \boxtimes W) \subset \Omega^{p+r, q+s} (X \times X , V \boxtimes W) .
\]

If $k \in \Omega^{(0,q), (n, n-s)} (X \times X , E \boxtimes E^*)$, then define the operator $\ul{k} : \Omega^{0,s}_c(X , E) \to \Omega^{0,q}(X, E)$ by the formula
\[
(\ul{k} \alpha)(x) = \int_{y \in X} k(x, y) \alpha(y) .
\]
We call $k$ the kernel of the smoothing operator $\ul{k}$.

For an open set $W \subset X \times X$, denote by 
\[
\sR^{r}(W , E \boxtimes E^*) \subset \Omega^{(0,\bu), (n, \bu)} (W \setminus \Delta , E \boxtimes E^*)
\]
the subspace of forms of type $(0,\bu) \times (n,\bu)$ which are regular of order $r$. 
Note that 
\[
\sR^{r}(W , E \boxtimes E^*) \subset \sR^{r+1}(W , E \boxtimes E^*) .
\]
Also, note that $\sR^{r}(W , E \boxtimes E^*)$ is {\em not} a chain complex since it increases order
\[
\dbar : \sR^{r}(W , E \boxtimes E^*) \to \sR^{r+1}(W , E \boxtimes E^*) .
\]

\begin{eg}
The Bochner--Martinelli kernel $\omega_{\rm \CC^n} \in \Omega^{(0,n-1), (n, n-1)}(\CC^n \times \CC^n \setminus \Delta)$ is regular of order $-1$:
\[
\omega_{\rm \CC^n} \in \sR^{-1}(\CC^n \times \CC^n) .
\]
\end{eg}

Of course, given any open subset of $\CC^n$ we can restrict the Bochner--Martinelli kernel to get a fundamental solution of $\dbar$ on the open neighborhood. 
Slightly more generally, there is the following.

\begin{lem}
Let $U \subset \CC^n$ be an open set, and suppose $V_0$ is any complex vector space.
Then
\[
\omega_{\CC^n}|_{U} \otimes {\rm Id}_V \in \sR^{-1}(U \times U , V_0 \otimes V_0^*)
\]
is a fundamental solution for $\dbar$ on $\Omega^{0,\bu}(U) \otimes V_0$. 
\end{lem}

Here is the general procedure.

Let $\sU = \{U_\alpha\}$ with coordinate maps $\{\varphi_\alpha : U_\alpha \to \CC^n\}$. 
Set $W_\alpha = U_\alpha \times U_\alpha$. 
Let 
\[
\omega_\alpha = (\varphi_\alpha \times \varphi_\alpha)^* \omega_{\CC^n} \in \sR^{-1}(W_\alpha) .
\]

\begin{dfn}
A \v{C}ech parametrix is a sequence $\vomega = (\omega^{(0)}, \ldots, \omega^{(n-1)})$ where:
\begin{itemize}
\item $\omega_{\alpha_0 \cdots \alpha_r}^{(r)} \in \sR^{-r-1}(W_{\alpha_0 \cdots \alpha_r})$ for $W_{\alpha_0 \cdots \alpha_{r}}$ a neighborhood of $\Delta \subset X \times X$
\end{itemize}
which together satisfy the equation
\[
(\delta \omega^{(r-1)})_{\alpha_0 \cdots \alpha_{r}} |_{W_{\alpha_0 \cdots \alpha_{r}}} = \dbar \omega^{(r)}_{\alpha_0\cdots \alpha_{r}}
\]
for each $1 \leq r \leq n-1$.
\end{dfn}

Given a \v{C}ech parametrix $\vomega$, define
\[
\omega^{(n)} = \delta \omega^{(n-1)} .
\]

We obtain a global solution as follows.
Let $\rho_\alpha \in C^\infty_c(W_\alpha)$ be a collection of compactly supported maps satisfying the two conditions:
\begin{itemize}
\item $\sum_{\alpha} \rho_\alpha = 1$ in some neighborhood of $\Delta \subset X \times X$;
\item ${\rm supp}(\rho_{\alpha_0} \cdots \rho_{\alpha_r}) \subset W_{\alpha_0 \cdots \alpha_r}$. 
\end{itemize}
Define the graded linear map
\[
\sD : \prod_{\alpha_0 \cdots \alpha_r} A^{n,\bu}(W_{\alpha_0 \cdots \alpha_r} - \Delta) \to \Omega^{n, \bu + r} (X \times X - \Delta)
\]
by
\[
\sD (\sigma^{(r)}) = \sum_{\alpha_0 \cdots \alpha_r} \rho_{\alpha_r} \dbar \rho_{\alpha_{r-1}} \cdots \dbar \rho_{\alpha_0} \sigma^{(r)}_{\alpha_0 \cdots \alpha_r} .
\]

\begin{thm}
Let $\vomega$ be a \v{C}ech parametrix on $X$ and $\{\rho_\alpha\}$ a collection of compactly supported maps as above.
Then
\[
\omega_X = \sum_{r = 0}^{n-1} (-1)^r \sD(\omega^{(r)}) \in \sR^{-1}(X \times X , E \boxtimes E^*)
\]
is a parametrix for the $\dbar$ on $X$:
\[
\dbar \ul{\omega}_X + \ul{\omega}_X \dbar = 1 - \ul{s}
\]
where $\ul{s}$ is a smoothing operator whose kernel satisfies
\[
s = \sD(\omega^{(n)})
\]
in some neighborhood of $\Delta \subset X \times X$.
\end{thm}

\section{Preliminary definitions}

Let $(E, Q, \omega)$ be the data of a free BV theory on a manifold $X$.
The sheaf of local functionals $\oloc(\sE)$ is equipped with the BV bracket $\{-,-\}$ of cohomological degree $+1$ and a differential induced by $Q$ which we denote by the same letter.
We also fix an open cover $\sU$ for $X$. 

Consider the \v{C}ech complex $\vC^\bu(\sU, \oloc(\sE))$ of the sheaf of local functionals. 
Recall that $\oloc(\cE)$ has an internal cohomological degree and differential $Q$. 
The complex we consider is the totalization of the bigraded complex consisting of the \v{C}ech differential $\delta$ and the differential $Q$. 

Together with the cup product on the \v{C}ech complex, the BV bracket endows $\vC^\bu(\sU, \oloc(\sE))$
with a bracket that we denote by $\{-,-\}^{\vee}$. 
For fixed \v{C}ech degrees $k, \ell$, it is defined by the composition
\[
\vC^k(\sU, \oloc(\sE)) \otimes \vC^\ell (\sU, \oloc(\sE)) \xto{\cup} \vC^{k + \ell} (\sU, \oloc(\sE) \otimes \oloc(\sE)) \xto{\{-,-\}} \vC^{k + \ell} (\sU, \oloc(\sE) \otimes \oloc(\sE))
\]
Explicitly, if $I$ is a $k$-cochain and $J$ is an $\ell$-cochain, then $\{I,J\}$ is the $(k+\ell)$-cochain defined by
\[
\{I,J\}^\vee_{\alpha_0 \cdots \alpha_{k+\ell}} = \{I_{\alpha_0 \cdots \alpha_k}, J_{\alpha_k \cdots \alpha_{k + \ell}}\}
\]
where the right-hand side is the usual BV bracket.

The bracket $\{-,-\}$ endows the shift of local functionals $\oloc (\sE)[-1]$ with the structure of a sheaf of dg Lie algebras, where the differential is given by $Q$. 
Similarly, $\{-,-\}^{\vee}$ gives $\vC^\bu(\sU, \oloc(\sE))[-1]$ the structure of a dg Lie algebra.

\begin{lem}
For any open cover $\sU$ the complex $\vC^\bu(\sU, \oloc(\sE))[-1]$ is a dg Lie algebra where the differential is $\delta + Q$ and the bracket is $\{-,-\}^\vee$. 
\end{lem}

\begin{dfn}
A {\bf \v{C}ech local functional} for an open cover $\sU$ of $X$ is an element 
\[
\vI \in \vC^\bu (\sU , \oloc(\sE)) .
\]
Suppose $\vI$ is of cohomological degree zero.
The {\bf \v{C}ech classical master equation} is the Maurer-Cartan equation
\[
(\delta + Q) \vI + \frac{1}{2} \{\vI,\vI\}^{\vee} = 0 .
\] 
\end{dfn}

A \v{C}ech local functional $\vI$ of degree zero determines a degree $+1$ element in the dg Lie algebra $\vC^\bu(\sU, \oloc(\sE))[-1]$. 
The condition that this local functional satisfy the \v{C}ech classical master equation is equivalent to the condition that it is a Maurer-Cartan element. 

\begin{dfn}
A {\bf \v{C}ech classical field theory on} $X$ for the open cover $\cU$ is a free BV theory $(E, Q, \omega)$ on $X$ together with a degree zero \v{C}ech local functional $\vI$ satisfying the \v{C}ech classical master equation. 
\end{dfn}

Fix a partition of unity $\rho = \{\rho_\alpha\}$ subordinate to the cover $\sU$ of $X$. 
For any sheaf $\cF$ on $X$ we define the map
\[
\sD_{\rho} : \vC^\bu(\sU , \sF) \to \sF(X) 
\]
by \brian{?}

\begin{lem}
The map of cochain complexes
\[
\sD_{\rho} : \vC^\bu(\sU, \oloc(\sE))[-1] \to \oloc(\sE)(X)[-1]
\]
is a map of dg Lie algebras. 
In particular, every \v{C}ech local functional $\vI$ satisfying the \v{C}ech master equation defines a local functional $I_{\sU, \rho} = \sD_\rho(\vI)$ which satisfies the usual classical master equation on $X$. 
\end{lem}

As an immediate corollary, we see that once we choose a partition of unity, every \v{C}ech classical theory $(E, Q, \omega, \vI)$ on $X$ defines an ordinary BV theory where the local interaction is given by $I_{\cU, \rho} = \sD_{\rho} (\vI)$. 

\begin{lem}
\brian{How does $I_{\cU,\rho}$ depend on the cover and $\rho$.}
\end{lem}

In fact, for nice enough covers even more is true.
\brian{how to say this} $\sU$ is \brian{nice enough} the map $\sD$ is a quasi-isomorphism of dg Lie algebras.
In this case, we see that there is an equivalence between \v{C}ech local functionals satisfying the \v{C}ech master equation and local functionals satisfying the usual master equation. 


\end{document}