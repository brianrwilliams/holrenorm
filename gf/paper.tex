% Created 2020-10-14 Wed 14:44
% Intended LaTeX compiler: pdflatex
\documentclass[11pt]{article}
\usepackage[utf8]{inputenc}
\usepackage[T1]{fontenc}
\usepackage{graphicx}
\usepackage{grffile}
\usepackage{longtable}
\usepackage{wrapfig}
\usepackage{rotating}
\usepackage[normalem]{ulem}
\usepackage{amsmath}
\usepackage{textcomp}
\usepackage{amssymb}
\usepackage{capt-of}
\usepackage{hyperref}
\usepackage{macros}
\author{John Doe}
\date{\today}
\title{On the local cohomology of holomorphic vector fields}
\hypersetup{
 pdfauthor={John Doe},
 pdftitle={On the local cohomology of holomorphic vector fields},
 pdfkeywords={},
 pdfsubject={},
 pdfcreator={Emacs 27.1 (Org mode 9.4)}, 
 pdflang={English}}
\begin{document}

\maketitle
\tableofcontents


\section{Introduction}
\label{sec:org0a5762e}
The Lie algebra cohomology of vector fields on a smooth oriented compact manifold are \brian\{something like: is a related cochain complex that has been studied extensively in the context of characteristic classes of foliations \cite{Fuks, Guillemin, LosikDiag, Bernstein}.\}

The Lie algebra cohomology of vector fields is, by definition, the cohomology of the Chevalley--Eilenberg cochain complex \(\clie^\bu(\Vect(M))\).
An important step in the computation of this cohomology is a computation of the cohomology of the \{\em diagonal subcomplex\} \(\clie^\bu_{\triangle} (\Vect(M)) \subset \clie^\bu(\Vect(M))\) which consists of cochains \(\varphi \in \clie^k(\Vect(M))\) satisfying \(\varphi(X_1,\ldots,X_k) = 0\) if \(\bigcap_{i=1}^k {\rm Supp}(X_i) = \emptyset\).

This paper concerns the cohomology of a smaller subcomplex of \{\em local cochains\} of \({\rm Vect}(M)\).
Roughly, a cochain \(\varphi \in \clie^k(\Vect(M))\) is local if it can be written as
\beqn
\(\varphi\) (X\textsubscript{1,\dots{}}, X\textsubscript{n}) = \(\int\)\textsubscript{M} L(X\textsubscript{1,\ldots}, X\textsubscript{n}) .
\eeqn
Here, \(L\) is a Lagrangian density, meaning it is a graded-symmetric polydifferential operator
\beqn
L : \Vect(M)\textsuperscript{\(\otimes\) k} \(\to\) \(\Omega\)\textsuperscript{\rm top}(M) .
\eeqn
The local cochain complex of \(\Vect(M)\) will be denoted \(\cloc^\bu(\Vect(M))\).
It is immediate to see that every local cochain is diagonal, so that there is a sequence of inclusions of cochain complexes
\beqn
\cloc\^{}\bu(\Vect(M)) \hookrightarrow \{\rm C\}\^{}\bu\textsubscript{\triangle} (\Vect(M)) \hookrightarrow \clie\^{}\bu(\Vect(M)) .
\eeqn

The local cohomology of vector fields is motivated, in part, by variational calculus, particularly as it pertains to classical and quantum field theory.
A classical field theory is, in part, prescribed by a Lagrangian density depending on the fields.
To describe the dynamics of a system one finds the extrema of the action functional given by integrating the Lagrangian density over the spacetime manifold.
In other words, the key piece of data is that of a local functional.

The theory of anomalies is an important structural aspect of quantum field theory.
In short, an anomaly describes the failure of a symmetry in a classical field theory to persist to a symmetry at the quantum level.
Similarly to action functionals, anomalies can be realized as local functionals.

A diffeomorphism invariant field theory on a manifold \(M\) receives an infinitesimal action by the Lie algebra of vector fields \(\Vect(M)\).
Anomalies for this infinitesimal action to exist at the quantum level are generally given by local cocycles in \(\cloc^\bu(\Vect(M))\).

\section{Definitions and main results}

In this section, \(X\) denotes a smooth manifold.
We work in the \$C\^{}\(\infty\)\$-category, so unless otherwise specified a ``section`` means a smooth section,
a ``differential form'' means a smooth differential form, etc..

Let \(E \to X\) denote a \$\ZZ\$-graded vector bundle on \(X\) and denote by \(\cE\) its sheaf of sections.
We consider the pro vector bundle of \$\(\infty\)\$-jets which we will denote by \(\jet (E)\), see \cite{Anderson} or \cite[\S 5.6]{CostelloBook} for instance.
The sheaf of smooth sections of this pro vector bundle carries the natural structure of a \$D\textsubscript{X}\$-module.

\begin{dfn}
Let $E$ be a graded vector bundle on $X$.
The sheaf of \defterm{Lagrangians} on $E$ is the $C^\infty_X$-module
\beqn
\Lag (E) \define \prod_{n > 0} {\rm Hom}_{C^\infty_X} \left(\jet (E) , C^\infty_X\right) .
\eeqn
\end{dfn}

\begin{rmk}
The notation ${\rm Hom}_{C^\infty_X} \left(\jet (E) , C^\infty_X\right)$ refers to the sheaf of continuous linear maps of $C^\infty_X$-modules.
This can be viewed as an ind vector bundle formally dual to the pro vector bundle $\jet (E)$.
The flat connection defining the $D_X$-module structure on $\jet (E)$ endows this sheaf with the structure of a $D_X$-module.
Notice that the constant functionals on $\jet (E)$ do not appear in the definition of $\Lag(E)$, this is mostly for conventional reasons and will not play a huge role in what follows.
\end{rmk}

For any graded vector bundle \(E\), the \$C\^{}\(\infty\)\textsubscript{X}\$-modules \(\Lag(E)\) has the natural structure of a \$D\textsubscript{X}\$-algebra, induced from the \$D\textsubscript{X}\$-module structure on \(\jet(E)\).

Let \({\rm Dens}_X\) be the right \$D\textsubscript{X}\$-module of densities on \(X\).
Given any left \$D\textsubscript{X}\$-module \(V\) one can consider the following sheaf
\beqn
\{\rm Dens\}\textsubscript{X} \(\otimes\)\textsubscript{D\textsubscript{X}} V .
\eeqn
If \(X\) is an oriented smooth manifold and \(V\) is flat, then this agrees with (a shift of) the de Rham complex of \(V\), see Remark \ref{rmk:derham} below.
For the case at hand, \(V\) is the left \$D\textsubscript{X}\$-module of Lagrangians \({\rm Lag} (E)\) and we have the following definition.

\begin{dfn}
Let $E$ be a vector bundle on $X$.
The sheaf of \defterm{local functionals} on $X$ is
\beqn
\oloc(E) \define {\rm Dens}_X \otimes_{D_X} {\rm Lag} (E) .
\eeqn
\end{dfn}

Concretely, a section of \({\rm Lag}(E)\) is a sum of functionals of the form
\beqn
\(\phi\) \(\in\) \cE \mapsto D\textsubscript{1} \(\phi\)\textsubscript{1} \(\cdots{}\)  D\textsubscript{n} \(\phi\)\textsubscript{n}
\eeqn
where \(D_i\) are differential operators acting on the bundle \(E\).
Likewise, a section of \(\oloc(E)\) is given as a sum of functionals which send a section \(\phi\) to a class
\beqn
\bigg[D\textsubscript{1} \(\phi\)\textsubscript{1} \(\cdots{}\)  D\textsubscript{n} \(\phi\)\textsubscript{n} \(\omega\) \bigg]
\eeqn
where \(\omega\) is a density on \(X\).
The brackets denotes an equivalence class where two sections are equivalent if they differ up to a total derivative.
For this reason, we will often write such an element using the integration symbol
\beqn
\(\int\) D\textsubscript{1} \(\phi\)\textsubscript{1} \(\cdots{}\)  D\textsubscript{n} \(\phi\)\textsubscript{n} \(\omega\)
\eeqn
where we provide the warning that no actual integration is occurring. \footnote{Of course, unless the section $\phi$ is compactly supported integration over an open subset is ill-defined.}

\begin{rmk}\label{rmk:derham}
If $X$ is an oriented smooth manifold, the sheaf of local functionals of $E$ can be expressed using the de Rham complex of the $D_X$-module of Lagrangians.
In this case, ${\rm Dens}_X$ can be replaced by the bundle of top forms $\Omega^{d}_X$ where $d = \dim_{\RR}(X)$.
This right $D_X$-module $\Omega^d_X$ has a free resolution of the form
\beqn
\Omega^0 \otimes_{C^\infty_X} D_X [d] \to \cdots \to \Omega^{d-1}_X \otimes_{C^\infty_X} D_X [1] \to \Omega^d_X \otimes_{C^\infty_X} D_X .
\eeqn
Since ${\rm Lag}(E)$ is flat as a $D_X$-module one can use this resolution to obtain a quasi-isomorphism
\beqn\label{derham1}
\oloc(E) \; \simeq \; \Omega^\bu \bigg( X \; , \; {\rm Lag}(E) \bigg)[d] .
\eeqn
We will use this description extensively throughout this paper.
For more details see Theorem \cite[Lemma 3.5.4.1]{CG2}.
In the unoriented case one would need to use a twisted version of the de Rham complex.
\end{rmk}

The next definition we will need is that of a local dg Lie algebra.
Roughly, this is a vector bundle whose sheaf of sections is equipped with a sufficiently well-behaved dg Lie algebra structure.

\begin{dfn}
A \defterm{local dg Lie algebra} on a smooth manifold $X$ is a triple $(L, \d, [\cdot , \cdot])$ where:
\begin{itemize}
\item[(i)] $L$ a $\ZZ$-graded vector bundle on $X$ of finite total rank;
\item[(ii)] $\d$ is a degree $+1$ differential operator $\d : \cL \to \cL$ on the sheaf $\cL$ of smooth sections of~$L$, and
\item[(iii)] $[\cdot, \cdot]$ is a bilinear polydifferential operator
\beqn
[\cdot , \cdot] : \cL \times \cL \to \cL
\eeqn
\end{itemize}
such that the triple $(\cL, \d, [\cdot,\cdot])$ carries the structure of a sheaf of dg Lie algebras.
\end{dfn}

Just as in the case of an ordinary graded vector bundle, we can discuss the Lagrangians on a local Lie algebra \(L\).
In this case, \({\rm Lag}(L)\) is equipped with the Chevalley--Eilenberg differential \(\d_{\rm CE}\) induced from the Lie algebra structure on \(L\).
In fact, the \$\(\infty\)\$-jet bundle \(\jet(L)\) is a dg Lie algebra object in \$D\textsubscript{X}\$-modules and we have the dg \$D\textsubscript{X}\$-module of reduced Chevalley-Eilenberg cochains
\beqn
\cred\^{}\bu (\jet(L)) = (\{\rm Lag\}(L), \d\textsubscript{\rm CE}) .
\eeqn
(Notice we look at reduced cochains since we have thrown out the constant functions in the definition of \({\rm Lag}(L)\).)
Since \(\d_{\rm CE}\) is compatible with the \$D\textsubscript{X}\$-module structure, this induces a differential on the space of local functionals \(\oloc(L)\).

We arrive at the central object of study of this paper.

\begin{dfn}
The \defterm{local Chevalley--Eilenberg cochain complex} of a local Lie algebra $\cL$ is the sheaf of cochain complexes
\begin{align}
\cloc^\bu(\cL) & \define \left(\oloc(L) , \d_{\rm CE} \right) \\ & = {\rm Dens}_X \otimes_{D_X} \cred^\bu(\jet(L)).
\end{align}
\end{dfn}

We now turn to the local Lie algebras of vector fields.
Throughout this paper we will focus mostly on the case of complex manifolds and holomorphic vector fields.
This is mostly for sake of applications to physics, see \S \ref{sec:applications}.
We remark on the smooth case (and other variants of vector fields) in \S \ref{sec:variants}.

\begin{eg}\label{eg:localT}
Let $X$ be a complex manifold and denote by $\tangent = \tangent X$ the holomorphic tangent bundle.
Consider its Dolbeault complex
\beqn
\cT \define \Omega^{0,\bu}(X , \tangent)
\eeqn
This is a sheaf of cochain complexes (in fact, it is an elliptic complex) where the differential is the $\dbar$-operator.
Moreover, this sheaf of cochain complexes is equipped with a bracket $[\cdot, \cdot]$ which extends the Lie bracket of vector fields.
This endows $\cT$ with the structure of a local Lie algebra.
\end{eg}

\begin{rmk}
The Dolbeault complex of any holomorphic vector bundle is a resolution for its sheaf of holomorphic sections.
Note that the sheaf of holomorphic vector fields is {\em not} a local Lie algebra since it is not given as the $C^\infty$-sections of a vector bundle.
Therefore, to capture the notion of holomorphic vector fields using local Lie algebras it is necessary to consider this resolution $\cT$.
Indeed, if $\cT^{\rm hol}$ denotes the sheaf of holomorphic vector fields, the embedding $\cT^{\rm hol} \hookrightarrow \cT$ is a quasi-isomorphism.
\end{rmk}

Our first main result pertains to the local cohomology of holomorphic vector fields and is the content of \S \ref{sec:global}.
To state the result, we introduce the notation \(\fw_n\) for the Lie algebra of holomorphic vector fields on the formal \$n\$-disk \(\Hat{D}^n\) as studied by \cite{GF}.

\begin{thm} \label{thm:global}
Let $X$ be a complex manifold of complex dimension $n$.
Then
\beqn
{\rm H}^k_{\rm loc}(\cT(X)) \; \cong \; \bigoplus_{i=0}^{2n} {\rm H}^i_{\rm dR}(X) \tensor {\rm H}^{2n + k-i}_{\rm red}(\fw_n) .
\eeqn
%In particular, if the manifold is connected the space of anomalies for holomorphic diffeomorphisms for a theory defined on $X$ is:
%\beqn
%H^{1}_{\rm loc}(\sT_X) =  H^{2d+1}_{\rm Lie}(\W_d)  ,
%\eeqn
%which is independent of the complex manifold.
\end{thm}

\begin{rmk}
The notation ${\rm H}_{\rm red}^{\bu} (\fw_n)$ refers to the reduced continuous cohomology of the Lie algebra of formal vector fields as pioneered by Gelfand and Fuks \cite{GF, Fuks}.
%This is the cohomology of the complex \brian{finish}
%The notation \brian{reduced}
\end{rmk}

\begin{rmk}
\brian{intuitive explanation for the decoupling of de Rham and Lie algebra cohomology}
\end{rmk}

In \S \ref{sec:descent} we specialize to the flat case \(X = \CC^n\) where the local cohomology reduces to a shift of the Gelfand--Fuks cohomology
\beqn
\{\rm H\}\^{}\bu\textsubscript{\rm loc}(\cT(\CC\textsuperscript{n})) \(\cong\) \{\rm H\}\textsubscript{\rm red}\^{}\bu(\fw\textsubscript{n})[2n]
\eeqn
by the first result.
In this case, we will describe an explicit quasi--isomorphism
\beqn\label{delta}
\(\delta\) : \clie\textsubscript{\rm red}\^{}\bu(\fw\textsubscript{n}) \xto{\simeq} \cloc\^{}\bu(\cT(\CC\textsuperscript{n})) .
\eeqn
The map \(\delta\) is constructed using the method reminiscent of ``topological descent`` \brian{refs}.
It utilizes the existence of two classes of degree \((-1)\) endomorphisms on the complex of local functionals that we denote \(\eta_i\) and \(\Bar{\eta}_i\).

These operators can be described heuristically as follows.
On \(\CC^n\), the local cochain complex \(\cloc^\bu (\cT(\CC^n))\) receives an action by the Lie algebra of translations spanned by the constant vector fields \(\partial_{z_i}\) and \(\partial_{\zbar_i}\).
The action of this Lie algebra is homotopically trivial.
The operator \(\eta_i\) provides an explicit trivialization for the action of the holomorphic vector field \(\partial_{z_i}\) and \(\Bar{\eta}_i\) provides a trivialization for \(\partial_{\zbar_i}\).

Using these homotopies, we can give a description of the map \(\delta\) in (\ref{delta}).
Notice that there is a map of Lie algebras \(j : \cT^{\rm hol}(D^n) \to \fw_n\) which records the Taylor expansion of a vector field at \(0 \in D^n\).
Here, \(\cT^{\rm hol}(D^n)\) denotes the Lie algebra of holomorphic vector fields on an \$n\$-disk centered at the origin.

\begin{thm}
When $X = \CC^n$, the quasi--isomorphism $\delta$ is defined by $\delta(\phi) = \int \phi^{n,n}$ where $\phi^{n,n}$ is the $\d^n z \d^n \zbar$-component of the expression
\beqn
\exp\left(\sum_{i=1}^n \left(\d \zbar_i \Bar{\eta}_i + \d z_i \eta_i\right)\right) j^*\phi .
\eeqn
%Here:
%\begin{itemize}
%\item $j^* \phi$ is the pullback of the class $\phi$ along the Taylor expansion of a holomorphic vector field, and
%\item the operators $\eta_i$, $\Bar{\eta}_i$ are degree one operators defined in Equations (\ref{eqn:holdescent}), (\ref{eqn:cartandescent}).
%\end{itemize}
\end{thm}

\begin{eg}
\brian{do an example on $\CC^2$.}
\end{eg}

\%\subsection{Variants of the main results}

\$\begin{itemize}
\%\item Symplectic vector fields.
\%\item Divergence free vector fields.
\%\end{itemize}
\%
\%\subsubsection{Super vector fields}
\%
\%\subsubsection{Smooth vector fields}
\%
\%\subsection{Applications} \label{sec:applications}
\end{document}
