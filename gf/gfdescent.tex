\documentclass[11pt]{amsart}

\pdfoutput = 1

\usepackage{macros,slashed,amsaddr}

\linespread{1.25}

\setcounter{tocdepth}{2}
\numberwithin{equation}{section}
\newcommand{\nocontentsline}[3]{}
\newcommand{\tocless}[2]{\bgroup\let\addcontentsline=\nocontentsline#1{#2}\egroup}
\newcommand{\changelocaltocdepth}[1]{%
  \addtocontents{toc}{\protect\setcounter{tocdepth}{#1}}%
  \setcounter{tocdepth}{#1}%
}
\setcounter{tocdepth}{1}

\def\brian{\textcolor{blue}{BW: }\textcolor{blue}}
\def\ingmar#1{{\textcolor{green!65!black}{IAS: {#1}}}}

\def\Dol{{\rm Dol}}
\def\dR{{\rm dR}}
\def\ep{{\varepsilon}}
\def\define{\overset{\rm def}{=}}
\newcommand{\defterm}[1]{\textbf{\emph{#1}}}
\def\Lag{{\rm Lag}}
\def\jet{{\rm J}}
\def\tangent{{\rm T}}
\def\desc{{\rm desc}}
\def\fgl{\mathfrak{gl}}

\addbibresource{descent.bib}

\begin{document}

\title{On the local cohomology of holomorphic vector fields}

\author{Brian R. Williams}
\address{School of Mathematics\\
University of Edinburgh \\ 
Edinburgh \\ 
UK}
\email{brian.williams@ed.ac.uk}

\maketitle

\tableofcontents

\section{Introduction}

The Lie algebra cohomology of vector fields on a smooth oriented compact manifold are \brian{something like: is a related cochain complex that has been studied extensively in the context of characteristic classes of foliations \cite{Fuks, Guillemin, LosikDiag, Bernstein}.}

The Lie algebra cohomology of vector fields is, by definition, the cohomology of the Chevalley--Eilenberg cochain complex $\clie^\bu(\Vect(M))$.
An important step in the computation of this cohomology is a computation of the cohomology of the {\em diagonal subcomplex} $\clie^\bu_{\triangle} (\Vect(M)) \subset \clie^\bu(\Vect(M))$ which consists of cochains $\varphi \in \clie^k(\Vect(M))$ satisfying $\varphi(X_1,\ldots,X_k) = 0$ if $\bigcap_{i=1}^k {\rm Supp}(X_i) = \emptyset$. 

This paper concerns the cohomology of a smaller subcomplex of {\em local cochains} of ${\rm Vect}(M)$. 
Roughly, a cochain $\varphi \in \clie^k(\Vect(M))$ is local if it can be written as
\beqn
\varphi (X_1,\dots, X_n) = \int_M L(X_1,\ldots, X_n) .
\eeqn
Here, $L$ is a Lagrangian density, meaning it is a graded-symmetric polydifferential operator
\beqn
L : \Vect(M)^{\otimes k} \to \Omega^{\rm top}(M) .
\eeqn
The local cochain complex of $\Vect(M)$ will be denoted $\cloc^\bu(\Vect(M))$. 
It is immediate to see that every local cochain is diagonal, so that there is a sequence of inclusions of cochain complexes
\beqn
\cloc^\bu(\Vect(M)) \hookrightarrow {\rm C}^\bu_{\triangle} (\Vect(M)) \hookrightarrow \clie^\bu(\Vect(M)) .
\eeqn

The local cohomology of vector fields is motivated, in part, by variational calculus, particularly as it pertains to classical and quantum field theory. 
A classical field theory is, in part, prescribed by a Lagrangian density depending on the fields. 
To describe the dynamics of a system one finds the extrema of the action functional given by integrating the Lagrangian density over the spacetime manifold.
In other words, the key piece of data is that of a local functional.

The theory of anomalies is an important structural aspect of quantum field theory. 
In short, an anomaly describes the failure of a symmetry in a classical field theory to persist to a symmetry at the quantum level.
Similarly to action functionals, anomalies can be realized as local functionals. 

A diffeomorphism invariant field theory on a manifold $M$ receives an infinitesimal action by the Lie algebra of vector fields $\Vect(M)$. 
Anomalies for this infinitesimal action to exist at the quantum level are generally given by local cocycles in $\cloc^\bu(\Vect(M))$. 

\section{Definitions and main results}

In this section, $X$ denotes a smooth manifold. 
We work in the $C^\infty$-category, so unless otherwise specified a ``section" means a smooth section, 
a ``differential form" means a smooth differential form, etc..

Let $E \to X$ denote a $\ZZ$-graded vector bundle on $X$ and denote by $\cE$ its sheaf of sections. 
We consider the pro vector bundle of $\infty$-jets which we will denote by $\jet (E)$, see \cite{Anderson} or \cite[\S 5.6]{CostelloBook} for instance. 
The sheaf of smooth sections of this pro vector bundle carries the natural structure of a $D_X$-module. 

\begin{dfn}
Let $E$ be a graded vector bundle on $X$.
The sheaf of \defterm{Lagrangians} on $E$ is the $C^\infty_X$-module
\beqn
\Lag (E) \define \prod_{n > 0} {\rm Hom}_{C^\infty_X} \left(\jet (E) , C^\infty_X\right) .
\eeqn
\end{dfn}

\begin{rmk}
The notation ${\rm Hom}_{C^\infty_X} \left(\jet (E) , C^\infty_X\right)$ refers to the sheaf of continuous linear maps of $C^\infty_X$-modules. 
This can be viewed as an ind vector bundle formally dual to the pro vector bundle $\jet (E)$. 
The flat connection defining the $D_X$-module structure on $\jet (E)$ endows this sheaf with the structure of a $D_X$-module. 
Notice that the constant functionals on $\jet (E)$ do not appear in the definition of $\Lag(E)$, this is mostly for conventional reasons and will not play a huge role in what follows. 
\end{rmk}

For any graded vector bundle $E$, the $C^\infty_X$-modules $\Lag(E)$ has the natural structure of a $D_X$-algebra, induced from the $D_X$-module structure on $\jet(E)$. 

Let ${\rm Dens}_X$ be the right $D_X$-module of densities on $X$. 
Given any left $D_X$-module $V$ one can consider the following sheaf
\beqn
{\rm Dens}_X \otimes_{D_X} V .
\eeqn
If $X$ is an oriented smooth manifold and $V$ is flat, then this agrees with (a shift of) the de Rham complex of $V$, see Remark \ref{rmk:derham} below. 
For the case at hand, $V$ is the left $D_X$-module of Lagrangians ${\rm Lag} (E)$ and we have the following definition. 

\begin{dfn}
Let $E$ be a vector bundle on $X$. 
The sheaf of \defterm{local functionals} on $X$ is 
\beqn
\oloc(E) \define {\rm Dens}_X \otimes_{D_X} {\rm Lag} (E) .
\eeqn
\end{dfn}

Concretely, a section of ${\rm Lag}(E)$ is a sum of functionals of the form
\beqn
\phi \in \cE \mapsto D_{1} \phi_1 \cdots  D_{n} \phi_n
\eeqn
where $D_i$ are differential operators acting on the bundle $E$. 
Likewise, a section of $\oloc(E)$ is given as a sum of functionals which send a section $\phi$ to a class 
\beqn
\bigg[D_{1} \phi_1 \cdots  D_{n} \phi_n \omega \bigg]
\eeqn
where $\omega$ is a density on $X$. 
The brackets denotes an equivalence class where two sections are equivalent if they differ up to a total derivative. 
For this reason, we will often write such an element using the integration symbol 
\beqn
\int D_{1} \phi_1 \cdots  D_{n} \phi_n \omega
\eeqn
where we provide the warning that no actual integration is occurring. \footnote{Of course, unless the section $\phi$ is compactly supported integration over an open subset is ill-defined.}

\begin{rmk}\label{rmk:derham}
If $X$ is an oriented smooth manifold, the sheaf of local functionals of $E$ can be expressed using the de Rham complex of the $D_X$-module of Lagrangians.
In this case, ${\rm Dens}_X$ can be replaced by the bundle of top forms $\Omega^{d}_X$ where $d = \dim_{\RR}(X)$. 
This right $D_X$-module $\Omega^d_X$ has a free resolution of the form
\beqn
\Omega^0 \otimes_{C^\infty_X} D_X [d] \to \cdots \to \Omega^{d-1}_X \otimes_{C^\infty_X} D_X [1] \to \Omega^d_X \otimes_{C^\infty_X} D_X .
\eeqn
Since ${\rm Lag}(E)$ is flat as a $D_X$-module one can use this resolution to obtain a quasi-isomorphism
\beqn\label{derham1}
\oloc(E) \; \simeq \; \Omega^\bu \bigg( X \; , \; {\rm Lag}(E) \bigg)[d] .
\eeqn
We will use this description extensively throughout this paper.
For more details see Theorem \cite[Lemma 3.5.4.1]{CG2}. 
In the unoriented case one would need to use a twisted version of the de Rham complex. 
\end{rmk}

The next definition we will need is that of a local dg Lie algebra. 
Roughly, this is a vector bundle whose sheaf of sections is equipped with a sufficiently well-behaved dg Lie algebra structure.

\begin{dfn} 
A \defterm{local dg Lie algebra} on a smooth manifold $X$ is a triple $(L, \d, [\cdot , \cdot])$ where:
\begin{itemize}
\item[(i)] $L$ a $\ZZ$-graded vector bundle on $X$ of finite total rank;
\item[(ii)] $\d$ is a degree $+1$ differential operator $\d : \cL \to \cL$ on the sheaf $\cL$ of smooth sections of~$L$, and
\item[(iii)] $[\cdot, \cdot]$ is a bilinear polydifferential operator
\beqn
[\cdot , \cdot] : \cL \times \cL \to \cL
\eeqn
\end{itemize}
such that the triple $(\cL, \d, [\cdot,\cdot])$ carries the structure of a sheaf of dg Lie algebras. 
\end{dfn}

Just as in the case of an ordinary graded vector bundle, we can discuss the Lagrangians on a local Lie algebra $L$. 
In this case, ${\rm Lag}(L)$ is equipped with the Chevalley--Eilenberg differential $\d_{\rm CE}$ induced from the Lie algebra structure on $L$. 
In fact, the $\infty$-jet bundle $\jet(L)$ is a dg Lie algebra object in $D_X$-modules and we have the dg $D_X$-module of reduced Chevalley-Eilenberg cochains 
\beqn
\cred^\bu (\jet(L)) = ({\rm Lag}(L), \d_{\rm CE}) . 
\eeqn
(Notice we look at reduced cochains since we have thrown out the constant functions in the definition of ${\rm Lag}(L)$.)
Since $\d_{\rm CE}$ is compatible with the $D_X$-module structure, this induces a differential on the space of local functionals $\oloc(L)$. 

We arrive at the central object of study of this paper.

\begin{dfn}
The \defterm{local Chevalley--Eilenberg cochain complex} of a local Lie algebra $\cL$ is the sheaf of cochain complexes
\begin{align}
\cloc^\bu(\cL) & \define \left(\oloc(L) , \d_{\rm CE} \right) \\ & = {\rm Dens}_X \otimes_{D_X} \cred^\bu(\jet(L)).
\end{align}
\end{dfn}

We now turn to the local Lie algebras of vector fields. 
Throughout this paper we will focus mostly on the case of complex manifolds and holomorphic vector fields.
This is mostly for sake of applications to physics, see \S \ref{sec:applications}.
We remark on the smooth case (and other variants of vector fields) in \S \ref{sec:variants}. 

\begin{eg}\label{eg:localT}
Let $X$ be a complex manifold and denote by $\tangent = \tangent X$ the holomorphic tangent bundle. 
Consider its Dolbeault complex 
\beqn
\cT \define \Omega^{0,\bu}(X , \tangent)
\eeqn
This is a sheaf of cochain complexes (in fact, it is an elliptic complex) where the differential is the $\dbar$-operator. 
Moreover, this sheaf of cochain complexes is equipped with a bracket $[\cdot, \cdot]$ which extends the Lie bracket of vector fields. 
This endows $\cT$ with the structure of a local Lie algebra. 
\end{eg}

\begin{rmk}
The Dolbeault complex of any holomorphic vector bundle is a resolution for its sheaf of holomorphic sections. 
Note that the sheaf of holomorphic vector fields is {\em not} a local Lie algebra since it is not given as the $C^\infty$-sections of a vector bundle. 
Therefore, to capture the notion of holomorphic vector fields using local Lie algebras it is necessary to consider this resolution $\cT$. 
Indeed, if $\cT^{\rm hol}$ denotes the sheaf of holomorphic vector fields, the embedding $\cT^{\rm hol} \hookrightarrow \cT$ is a quasi-isomorphism. 
\end{rmk}

Our first main result pertains to the local cohomology of holomorphic vector fields and is the content of \S \ref{sec:global}. 
To state the result, we introduce the notation $\fw_n$ for the Lie algebra of holomorphic vector fields on the formal $n$-disk $\Hat{D}^n$ as studied by \cite{GF}. 

\begin{thm} \label{thm:global}
Let $X$ be a complex manifold of complex dimension $n$.
Then
\beqn
{\rm H}^k_{\rm loc}(\cT(X)) \; \cong \; \bigoplus_{i=0}^{2n} {\rm H}^i_{\rm dR}(X) \tensor {\rm H}^{2n + k-i}_{\rm red}(\fw_n) .
\eeqn
%In particular, if the manifold is connected the space of anomalies for holomorphic diffeomorphisms for a theory defined on $X$ is:
%\beqn
%H^{1}_{\rm loc}(\sT_X) =  H^{2d+1}_{\rm Lie}(\W_d)  ,
%\eeqn
%which is independent of the complex manifold.
\end{thm}

\begin{rmk}
The notation ${\rm H}_{\rm red}^{\bu} (\fw_n)$ refers to the reduced continuous cohomology of the Lie algebra of formal vector fields as pioneered by Gelfand and Fuks \cite{GF, Fuks}.
%This is the cohomology of the complex \brian{finish}
%The notation \brian{reduced}
\end{rmk}

\begin{rmk}
\brian{intuitive explanation for the decoupling of de Rham and Lie algebra cohomology}
\end{rmk}

In \S \ref{sec:descent} we specialize to the flat case $X = \CC^n$ where the local cohomology reduces to a shift of the Gelfand--Fuks cohomology 
\beqn
{\rm H}^\bu_{\rm loc}(\cT(\CC^n)) \cong {\rm H}_{\rm red}^\bu(\fw_n)[2n]
\eeqn
by the first result.
In this case, we will describe an explicit quasi--isomorphism
\beqn\label{delta}
\delta : \clie_{\rm red}^\bu(\fw_n) \xto{\simeq} \cloc^\bu(\cT(\CC^n)) .
\eeqn
The map $\delta$ is constructed using the method reminiscent of ``topological descent" \brian{refs}. 
It utilizes the existence of two classes of degree $(-1)$ endomorphisms on the complex of local functionals that we denote $\eta_i$ and $\Bar{\eta}_i$.

These operators can be described heuristically as follows. 
On $\CC^n$, the local cochain complex $\cloc^\bu (\cT(\CC^n))$ receives an action by the Lie algebra of translations spanned by the constant vector fields $\partial_{z_i}$ and $\partial_{\zbar_i}$. 
The action of this Lie algebra is homotopically trivial. 
The operator $\eta_i$ provides an explicit trivialization for the action of the holomorphic vector field $\partial_{z_i}$ and $\Bar{\eta}_i$ provides a trivialization for $\partial_{\zbar_i}$. 

Using these homotopies, we can give a description of the map $\delta$ in (\ref{delta}). 
Notice that there is a map of Lie algebras $j : \cT^{\rm hol}(D^n) \to \fw_n$ which records the Taylor expansion of a vector field at $0 \in D^n$. 
Here, $\cT^{\rm hol}(D^n)$ denotes the Lie algebra of holomorphic vector fields on an $n$-disk centered at the origin.

\begin{thm}
When $X = \CC^n$, the quasi--isomorphism $\delta$ is defined by $\delta(\phi) = \int \phi^{n,n}$ where $\phi^{n,n}$ is the $\d^n z \d^n \zbar$-component of the expression
\beqn
\exp\left(\sum_{i=1}^n \left(\d \zbar_i \Bar{\eta}_i + \d z_i \eta_i\right)\right) j^*\phi .
\eeqn
%Here:
%\begin{itemize}
%\item $j^* \phi$ is the pullback of the class $\phi$ along the Taylor expansion of a holomorphic vector field, and 
%\item the operators $\eta_i$, $\Bar{\eta}_i$ are degree one operators defined in Equations (\ref{eqn:holdescent}), (\ref{eqn:cartandescent}).
%\end{itemize}
\end{thm}

\begin{eg}
\brian{do an example on $\CC^2$.}
\end{eg}

%\subsection{Variants of the main results}

%\subsubsection{Subalgebras} 
%\begin{itemize}
%\item Symplectic vector fields. 
%\item Divergence free vector fields. 
%\end{itemize}
%
%\subsubsection{Super vector fields}
%
%\subsubsection{Smooth vector fields}
%
%\subsection{Applications} \label{sec:applications}

\section{Formal geometry}

The key idea involved in the proof of Theorem \ref{thm:global} is to describe the local cohomology of holomorphic vector fields using formal geometry. 
This type of formal geometry we attribute back to Gelfand and Kazhdan \cite{GelfandICM, GK} and has been used \brian{...}
\cite{BK}. 

For notations and setup most similar to the approach we take here we refer to \cite[Part 1]{GGW} or \cite{SiZhengpingKai}. 
Throughout this section $X$ is a complex manifold of complex dimension $n$. 
On $X$, there is the holomorphic principal $\GL_n$-bundle ${\rm Fr}_X$ of holomorphic $n$-frames. 

Any $\GL_n$-representation $V$ determines a holomorphic vector bundle
\beqn\label{borel}
V_X \define {\rm Fr}_X \times^{\GL_n} V
\eeqn
on $X$. 
Recall that we can write the space $\Omega^k(X, V_X)$ of $k$-forms valued in $V_X$ equivalently as the space of basic $k$-forms on ${\rm Fr}_X$:
\beqn
\Omega^k({\rm Fr}_X , V)_{\rm bas} = \left\{\alpha \in \left(\Omega^k({\rm Fr}_X) \otimes V\right)^{\GL_n} \; | \; \iota_{\xi_A} \alpha = 0 \; , \; {\rm for\;all} \; a \in \fgl_n \right\}.
\eeqn
We have denoted by $\xi_a$ the vertical holomorphic vector field on ${\rm Fr}_X$ corresponding to $A \in \fgl_n$. 

We assume, additionally, that $V$ has the compatible structure of a module for the Lie algebra $\fw_n$ of formal vector fields on the formal $n$-disk.
Here, compatible means the following.

After choosing a formal coordinate, we have an embedding of Lie algebras $i : \fgl_n \to \fw_n$, where the $n\times n$ matrix $(a_{ij})$ is realized by the vector field $\sum_{ij} a_{ij} z_i \partial_{z_j}$.
We require that the composition 
\[
\fgl_n \xto{i} \fw_n \xto{\rho_\fw} {\rm End}(V) 
\]
is equal to ${\rm Lie}(\rho_{\rm GL})$. 
Here $\rho_{\fw}$ denotes the action of $\fw_n$ of $\rho_{\rm GL}$ is the original action of $\GL_n$. 
Such a structure on $V$ is called a {\em Harish-Chandra} module for the pair $(\fw_n, \GL_n)$. 

The structure of a Harish--Chandra module on $V$ allows one to define a {\em flat connection} on the bundle $V_X$ in (\ref{borel}). 
In other words, $V_X$ carries the structure of a (smooth) $D_X$-module. 
This flat connection is defined using the following universal connection which is at the heart of the Gelfand--Kazhdan approach to forrmal geometry. 

There exists a $\GL_n$-invariant $\fw_n$-valued holomorphic one-form
\[
\omega_{\rm Groth} \in \Omega^{1, \rm hol}({\rm Fr}_X , \fw_n)
\]
satisfying:
\begin{enumerate}
\item \label{groth1} for all $A \in \fgl_n$, one has $\omega_{\rm Groth} (\xi_A) = A$;
\item \label{gorth2} $\omega_{\rm Groth}$ satisfies the Maurer--Cartan equation
\beqn\label{eqn:mc}
\partial \omega_{\rm Groth} + \frac12 [\omega_{\rm Groth}, \omega_{\rm Groth}] = 0 .
\eeqn
\end{enumerate}

\begin{rmk}
The one-form $\omega_{\rm Groth}$ comes from a connection on the holomorphic {\em coordinate bundle} of $X$. 
Roughly, the holomorphic coordinate bundle $X^{\rm coor}$ is a space over $X$ whose fiber over a point $x \in X$ consists of all 

Sometimes called the ``Grothendieck" connection, there is a $\fw_n$-valued one-form $\Tilde{\omega}_{\rm Groth}$ on $X^{\rm coor}$ satisfying the Maurer--Cartan equation.
\brian{finish}
\end{rmk}

Using $\omega_{\rm Groth}$, we consider the connection on $V_X$ defined by
\beqn
\nabla_V^{\rm flat} = \d + \rho_{\fw} (\omega_{\rm Groth}) .
\eeqn
By the Maurer--Cartan equation (\ref{eqn:mc}), this connection is flat. 
Moreover, since $V_X$ is a holomorphic bundle over $X$, we have a canonical quasi-isomorphism
\beqn
\left(\Omega^{\bu, \rm hol} (X , V_X) \; , \; \nabla_V \right) \xto{\simeq}\left(\Omega^{\bu, \bu} (X , V_X) \; , \; \nabla_V^{\rm flat} \right) 
\eeqn 
where $\nabla_V = \dbar -  \nabla_V^{\rm flat}$.
Here $\dbar$ is the $\dbar$-operator for $V_X$.
The operator $\nabla_V$ endows $V_X$ with the structure of a {\em holomorphic} $D^{\rm hol}_X$-module.

%\begin{lem}\label{lem:symm}
%The $\fg$-module structure on the cochain complexes $\clie^*(\fg)$ and $\cred^*(\fg)$ is homotopically trivial.
%\end{lem}


\section{Cohomology on a general manifold} \label{sec:global}

We will obtain Theorem \ref{thm:global} from a slightly more refined statement at the level of sheaves. 
For a vector space $F$, we denote by $\ul{F}$ the corresponding sheaf which assigns to every open set the value $F$. 
In this section, we prove the following.

\begin{thm}
Let $X$ be a complex manifold of dimension $n$ and consider the local Lie algebra $\cT$ given by the Dolbeault complex of the holomorphic tangent bundle. 
There is a quasi-isomorphism of sheaves of cochain complexes on $X$:
\beqn
\cloc^\bu(\cT) \; \simeq \; \ul{{\rm C}_{\rm red}^*(\fw_n)} [2n] .
\eeqn
\end{thm}

\begin{proof}
As recounted in Equation (\ref{derham1}), there is a quasi-isomorphism of the local cohomology of a local Lie algebra $\cL$ with the de Rham complex of the $D_X$-module ${\rm C}^\bu_{\rm red} (\jet (L))$.
Applied to the local Lie algebra $\cT$ on the complex $n$-fold $X$, this reads:
\beqn
\cloc^\bu(\cT) \; \simeq \; \Omega^\bu \bigg( X \; , \; {\rm C}^\bu_{\rm red} \left( \jet (\cT) \right) \bigg) [2n] .
\eeqn

%We have used the notation $JE$ to denote the smooth sections of the infinite rank vector bundle ${\rm Jet}(E)$. 
%If $E$ is a holomorphic vector bundle let ${\rm Jet}^{hol}(E)$ denote the infinite rank holomorphic vector bundle of holomorphic jets. 
%Similarly, let $J^{hol}E$ be the holomorphic sections of this bundle. 
%This is a $D^{hol}_X$-module where $D^{hol}_X$ is the sheaf of holomorphic differential operators. 
%Equivalently, a $D^{hol}_X$-module is a holomorphic vector bundle with a holomorphic flat connection.
%Of course, any $D_X^{hol}$-module $E$ forgets to a smooth $D_X$-module that we denote $E^{C^\infty}$. 

%Suppose $V$ is a $D_X^{hol}$-module, and consider its holomorphic de Rham complex $\Omega^{*,hol}(X, V)$. 
%Moreover, $V$ forgets down to a $D_X$-module that we denote $V^{C^\infty}$ and we can consider its $C^\infty$ de Rham complex $\Omega^*(X, V^{C^\infty})$. 
%These two de Rham complexes are quasi-isomorphic since they are resolutions of the same sheaf. 

%\begin{lem} \label{lem: hol dmodule}
%There is a quasi-isomorphism of de Rham complexes
%\beqn
%\Omega^*(X, E^
%\end{lem}
%\begin{proof}
%Let $\sL$ be the sheaf of sections of $L$. 
%The Dolbeualt complex is a resolution of the sheaf of holomorphic sections; thus there is a quasi-isomorphism $\sL \simeq \sL^{hol}$ of $\sO_X$-modules. 
%\end{proof}

%This means that we can further reduce the expression for the local cohomology in (\ref{de rham 1}) to 
%\be\label{de rham 2}
%\cloc^*(\sL) \simeq \Omega^*(X , \cred^*(J^{hol}L^{hol}))[2d] .
%\ee
%We have dropped the notation $(-)^{C^\infty}$ for convenience.

As sheaves, we know $\cT$ is a resolution for the sheaf of holomorphic vector fields on $X$. 
Similarly, there is a quasi-isomorphism of (smooth) $D_X$-modules $\jet(\cT) \simeq \jet^{\rm hol} (\tangent)$, where $\jet^{\rm hol}(\tangent)$ denotes the holomorphic bundle of holomorphic $\infty$-jets of the holomorphic tangent bundle. 
It follows that there is a quasi-isomorphism of de Rham complexes
\beqn\label{smoothhol}
\Omega^\bu \bigg( X \; , \; {\rm C}^\bu_{\rm red} \left( \jet (\cT) \right) \bigg) \; \simeq \; \Omega^\bu \bigg(X \; , \; \cred^\bu\left(\jet^{\rm hol} (\tangent) \right) \bigg) .
\eeqn
On the right-hand side we emphasize that we take {\em holomorphic} jets.

For $\cV$ a module for the pair $(\fw_n, \GL_n)$, Gelfand-Kazhdan descent along the complex manifold $X$ yields a $D^{\rm hol}_X$-module $\desc_X(\cV)$. 
In the case that $\cV = \Hat{\tangent}_n$ we have seen that the $D^{\rm hol}_X$-module $\desc_X(\Hat{\tangent}_n)$ is equivalent to the $D^{\rm hol}_X$-module $\jet^{\rm hol} (\tangent)$. 

We now consider the $(\fw_n, \GL_n)$-module $\cred^\bu(\fw_n)$. 
By Lemma \ref{lem:symm}, there is a string of isomorphisms of $D^{\rm hol}_X$-modules
\beqn
\jet^{\rm hol} \cred^\bu (\tangent) = \desc(\cred^\bu(\fw_n)) \cong \cred^\bu(\desc(\fw_n)) = \cred^\bu({\rm J}^{\rm hol} \tangent).
\eeqn
Alternatively, by Proposition A.2 of \cite{GG3} we know that the functor of jets is symmetric monoidal, so the same result follows.

To summarize, we see that the Gelfand-Kazhdan descent of the $(\fw_n, \GL_n)$-module $\cred^\bu(\fw_n)$ is equal to the $D_X$-module $\cred^\bu(\jet^{\rm hol} \tangent)$.
From (\ref{smoothhol}), this is precisely the $D_X$-module present in the definition of the local cohomology of $\cT$.

Combining these facts, we obtain a quasi-isomorphism of sheaves on $X$:
%\beqn
%\cloc^\bu(\cT) \simeq \Omega^\bu\bigg(X ,  \cred^\bu\left(\jet^{\rm hol}(\tangent) \right)\bigg) .
%\eeqn
%Thus, the de Rham complex of the $D_X$-module given by descent is precisely the local cohomology 
\beqn
\cloc^\bu(\cT) \; \simeq \; \Omega^\bu\bigg(X , \desc_X\left(\cred^\bu(\fw_n)\right)\bigg)
\eeqn

The interpretation via descent will allow us to describe this de Rham complex explicitly. 
Suppose that $\fg$ is any Lie algebra.
Then $\fg$ acts on itself (and its dual) via the adjoint action. 
This extends to an action of $\fg$ on its Chevalley-Eilenberg complex $\clie^\bu(\fg ; M)$, where $M$ is any $\fg$-module via the formula
\beqn\label{eqn:liederiv}
(L_x \varphi) (x_1,\ldots, x_k) = \sum_i \varphi(x_1,\ldots, [x, x_i], \ldots,x_k) - x \cdot \varphi(x_1,\ldots,x_k) .
\eeqn
Here, $x, x_i \in \fg$ and $\varphi$ is a $k$-cochain with values in $M$.
The $[-,-]$ denotes adjoint action, and the $\cdot$ is the $\fg$-module structure on $M$. 
The same formula holds for the reduced cochains.

The action by any fixed element $x\in \fg$ on $\clie^\bu(\fg ; M)$ can be trivialized in a standard way. 
Define the endomorphism $i_x$ of the cochain complex $\clie^\bu(\fg ; M)$ by the formula
\beqn
(i_x \varphi) (x_1,\ldots, x_{k-1}) = \sum_i \varphi (x_1,\ldots, x_i, x , x_{i+1}, \ldots, x_{k-1}) .
\eeqn
Then, at the level of Lie algebra cochains, Cartan's magic formula holds:
\beqn
[\d_{\rm CE}, i_x] = L_x
\eeqn
where $L_x$ denotes the action of $x \in \fg$ as in (\ref{eqn:liederiv}). 

Applied to the case at hand, we see that $\fw_n$ acts homotopically trivially on $\cred^\bu(\fw_n)$ which implies that the descent $\desc_X(\clie^\bu(\fw_n)) \cong \cred^\bu(\jet(\tangent))$ is equipped with a homotopically trivial $D_X$-module structure. 
Equivalently, this means that the flat connection on $\cred^\bu(\jet(\tangent))$ is gauge equivalent to the trivial connection.

Thus, there is a quasi-isomorphism of de Rham complexes
\beqn
\Omega^\bu\bigg(X ,  \cred^\bu\left(\jet^{\rm hol}(\tangent)\right)\bigg) \; \simeq \; \Omega^\bu_X \tensor_{C^\infty_X} \ul{\cred^\bu(\fw_n)}
\eeqn
where the underline denotes the trivial $C^\infty_X$-module with fiber $\cred^\bu(\fw_n)$.
Since $\Omega^\bu_X$ is a resolution for the trivial $C^\infty_X$-module $\ul{\CC}$, the result follows.
\end{proof}


\section{Descent equations and local representatives}\label{sec:descent}

The goal in this section is to describe an explicit representative for the quasi-isomorphism of Theorem \ref{thm:global} in the affine case $X = \CC^n$. 

The global sections on $\CC^n$ of the local dg Lie algebra $\cT$ is the dg Lie algebra $\cT(\CC^n) = \Omega^{0,\bu}(\CC^n, \tangent)$.
This is a resolution for the Lie algebra $\cT^{hol}(\CC^n)$ of holomorphic vector fields on $\CC^n$. 
In fact, the projection onto the zeroeth cohomology 
\beqn
p : \cT(\CC^n) \xto{\simeq}  H^0(\cT(\CC^n)) = \cT^{hol}(\CC^n) .
\eeqn
is a quasi-isomorphism.

There is a map of dg Lie algebras
\beqn
j_0^\infty :  \cT^{hol}(\CC^n) \to \fw_n
\eeqn
which takes the Taylor expansion of a holomorphic vector field at $0 \in \CC^n$. 
We denote the composition $j \define j_0^\infty \circ p : \cT(\CC^n) \to \fw_n$. 

The map $j$ defines a map on the (continuous) Chevalley--Eilenberg cochain complexes
\beqn
j^* : \clie^\bu(\fw_n) \to \clie^\bu\left(\cT(\CC^n))\right) .
\eeqn

%Of course, $\cT(\CC^n)$ is the global sections of the local Lie algebra $\cT$ on $\CC^n$, as defined in Example \ref{eg:localT}. 
Recall that associated to the local Lie algebra $\cT$ we have the $\infty$-jet bundle $\jet (\cT)$. 
We denote the dg Lie algebra of global sections of this jet bundle by $\jet (\cT) (\CC^n)$.
By construction, we note that the map $j^*$ factors through the embedding of cochain complexes
\beqn
\clie^\bu\bigg( \jet (\cT) (\CC^n) \bigg) \hookrightarrow \clie^\bu \left( \cT(\CC^n) \right) .
\eeqn

%As we defined in \S \ref{??}, as a graded vector space $\clie^{\#}(J \cT(\CC^n))$ is the space of global sections of a graded (infinite rank) vector bundle on $\CC^n$ of the form ${\rm Sym} \left((J \cT)^\vee [-1] \right)$. 
Equipped with the Chevalley--Eilenberg differential $\clie^\bu(\jet(\cT))$ becomes a complex of (infinite rank) vector bundles. 
So, we obtain for each $\phi \in \clie^\bu(\fw_n)$ a global section $j^* \phi$ of the vector bundle $\clie^\bu\left(\jet(\cT)\right)$. 

\begin{eg}
Suppose $n=1$ and consider the $1$-cochain $\phi : f(z) \frac{\partial}{\partial z} \mapsto f'(0)$ of $\fw_1$. 
The value of the section $j^* \phi$ at the point $z_0 \in \CC$ is the cochain for $\cT = \Omega^{0,\bu}(\CC, T_\CC)$ defined by
\beqn
a(z,\zbar) \frac{\partial}{\partial z} + b(z,\zbar) \d \zbar \frac{\partial}{\partial z} \mapsto \frac{\partial}{\partial z} a(z,\zbar) |_{z = z_0} .
\eeqn
\end{eg}

On any manifold, we have seen that $\clie^\bu\left(\jet(\cT)\right)$ is commutative dg algebra in the category of $D$-modules.
On $\CC^n$ we consider the associated de Rham complex
\beqn
\Omega^\bu \bigg( \CC^n \; , \; \clie^\bu \left( \jet (\cT) \right) \bigg)
\eeqn
Recall that up to a shift (and upon taking reduced cochains) this complex is quasi-isomorphic to the local cohomology of $\cT$. 

Via the map $j$, a cochain $\phi \in \clie^\bu(\fw_n)$ determines a zero form in this de Rham complex $\phi^0 \define j^* \phi \in \Omega^0\left(\CC^n , \clie^\bu(\jet(\cT)) \right)$.

We denote by $\d_{\rm dR}$ the de Rham differential on this de Rham complex and $\d_{\rm CE}$ the Chevalley--Eilenberg differential for the dg Lie algebra $\jet(\cT)$. 
In general, the section $\phi^0$ is not flat, but we have the following. 

\begin{thm}
\label{thm:gfdescent}
Suppose $\phi \in \clie^\bu(\fw_n)$ and let $\phi^0 = j^* \phi$. 
Then, there exists $\phi^{i,j} \in \Omega^{i,j} (\CC^n , \clie^\bu \left( \jet (\cT) \right))$, $1 \leq i,j \leq n$ such that the element 
\beqn
\Phi \define \sum_{i,j} \phi^{i,j} 
\eeqn
satisfies the equation $(\d_{\rm dR} + \d_{\cT}) \Phi = 0$. 
\end{thm}

Using the Hodge decomposition of the Rham differential $\d_{\rm dR} = \dbar + \partial$, we we will actually show that the elements $\phi^{i,j}$ satisfy a pair of descent equations:
\begin{itemize}
\item Holomorphic descent equations:
\beqn\label{eqn:holdescent}
\dbar \phi^{i,j} + \dbar_{\cT} \phi^{i, j+1} = 0
\eeqn
for $0 \leq i , j \leq n$.
Here $\dbar_{\cT}$ denotes the differential internal to the dg Lie algebra $\cT$. 
\item Cartan descent equations:
\beqn\label{eqn:cartandescent}
\partial \phi^{i,j} + \d_{{\rm CE}} \phi^{i+1, j} = 0
\eeqn
for $0 \leq i , j \leq n$. 
Here $\d_{\rm CE}$ denotes the differential associated to the Lie bracket for the dg Lie algebra $\cT$. 
\end{itemize}

In fact, the elements $\phi^{i,j}$ have explicit forms which we proceed to describe. 
Define the degree $(-1)$ derivation $\Bar{\eta}_i$ of the dg Lie algebra $\Omega^{0,\bu}(\CC^n, \tangent)$ by 
\beqn\label{eqn:holdescent}
\Bar{\eta}_i \left(\alpha(z,\zbar) \frac{\partial}{\partial z_j} \right) = \left(i_{\frac{\partial}{\partial \zbar_i}} \alpha \right) (z,\zbar) \frac{\partial}{\partial z_j} .
\eeqn
On the right-hand side, $i_X \alpha$ denotes the contraction of the differential form $\alpha$ by the vector field $X$. 
This derivation extends to a derivation of the algebra $\clie^\bu(\cT(\CC^n))$ that we denote by the same symbol. 

Next, define the derivation $\eta_i$ of the algebra $\clie^\bu(\cT(\CC^n))$ by the formula
\beqn\label{eqn:cartandescent}
\eta_i (\psi) = \iota_{\frac{\partial}{\partial z_i}} \psi .
\eeqn
The right-hand side is the contraction of the cochain $\psi \in \clie^\bu(\cT(\CC^n))$ by the vector field $\frac{\partial}{\partial z_i}$. 
Explicitly, if $\psi$ is $k$-linear, then 
\beqn
(\iota_{\frac{\partial}{\partial z_i}} \psi)(\xi_1,\ldots, \xi_{k-1}) = \sum_{j=1}^{k-1} (\pm) \psi\left(\xi_1,\ldots, \xi_j, \frac{\partial}{\partial z_i}, \xi_{j+1}, \ldots, \xi_{k-1} \right) .
\eeqn

Given this notation, we return to the starting data which is a cochain $\phi \in \clie^\bu(\fw_n)$.
Recall, we set $\phi^0 = j^* \phi$ which is a zero form in the de Rham complex of jets. 
A representative for $\Phi$ as in the theorem is
\beqn\label{eqn:Phi}
\Phi = \exp\left(\sum_{i=1}^n \left(\d \zbar_i \Bar{\eta}_i + \d z_i \eta_i\right)\right) \phi^0 . 
\eeqn
Note that the derivations $\eta_i$ and $\Bar{\eta}_j$ commute for all $i,j$, so the right-hand side of the equation is unambiguously defined. 

With the notation set up we can now provide the proof of \ref{thm:gfdescent}. 

\begin{proof}
The differential on the de Rham complex $\Omega^\bu(\CC^n , \clie^\bu(\cT))$ has the form $\d_{dR} + \d_{\cT}$ where $\d_{\rm dR}$ is the de Rham differential encoded by the flat connection on $\clie^\bu(\cT)$ and $\d_{\cT}$ is the differential internal to the complex $\clie^\bu(\cT)$. 
Note that $\d_{\cT}$ splits as $\d_{\cT} = \dbar_{\cT} + \d_{\rm CE}$ where $\dbar_{\cT}$ is the $\dbar$-operator arising in the definition of $\cT$ (we use this notation to not confuse it with the de Rham differential), and $\d_{\rm CE}$ is the differential arising from the Lie bracket on $\cT$. 
Further, the de Rham differential $\d_{\rm dR}$ decomposes as $\d_{\rm dR} = \dbar + \partial$. 

It suffices to show that the elements $\phi^{i,j}$ satisfy the pair of descent equations (\ref{eqn:holdescent}) and (\ref{eqn:cartandescent}). 
Since the operators $\eta_i$ and $\Bar{\eta}_i$ commute, it suffices to prove (\ref{eqn:holdescent}) for $i=0$ and (\ref{eqn:cartandescent}) for $j=0$. 

Note that $\phi^{0,j}$ is the $(0,j)$th component in the expansion of $\exp\left(\sum_\ell \d \zbar_\ell \Bar{\eta}_\ell \right) \phi^0$. 
For $i=0$, descent equation (\ref{eqn:holdescent}) follows from the string of equalities
\beqn
\dbar_{\cT} \d \zbar_\ell \Bar{\eta}_\ell \phi^{0, j} = - \d \zbar_\ell [\dbar_{\cT}, \Bar{\eta}_\ell] \phi^{0,j} = -\d \zbar_\ell \frac{\partial}{\partial \zbar_\ell} \phi^{0,j} .
\eeqn

For the second descent equation, note that $\phi^{i,0}$ is the $(i,0)$th component in the expansion $\exp\left(\sum_\ell \d z_\ell  \right) \phi^0$.
For $j=0$, descent equation (\ref{eqn:cartandescent}) follows from the string of equalities 
\beqn
\d_{\rm CE} \d z_\ell \eta_\ell \phi^{i,0} = - \d z_{\ell} [\d_{\rm CE}, \eta_\ell] \phi^{i,0} = - \d z_\ell \frac{\partial}{\partial z_\ell} \phi^{i,0} .
\eeqn
\end{proof}

Combining this result with Theorem \ref{thm:global} we obtain the following.

\begin{cor}
The composite map 
\beqn
\delta : \clie_{\rm red}^\bu(\fw_n) \to \Omega^\bu\left(\CC^n , \clie_{\rm red}^\bu(\cT) \right) \xto{\simeq} \cloc^\bu(\cT(\CC^n)) [-2n].
\eeqn
which sends $\phi \mapsto \delta(\phi) = \int \phi^{n,n}$ is a quasi-isomorphism.
In particular, if $\phi \in \clie^\bu(\fw_n)$ is a Gelfand--Fuks cocycle of degree $k$, then $\delta (\phi) \in \cloc^\bu(\cT)$ is a local cocycle of degree $k-2n$ and up to equivalence all such local cocycles are obtained in this way.
\end{cor}

\begin{eg}
The reduced cohomology of one-dimensional formal vector fields is one-dimensional concentrated in degree $+3$. 
A representative for this class is the Gelfand--Fuks cocycle $\phi \in \clie^3(\fw_1)$ defined by
\beqn
\phi \bigg(f(t) \frac{\d}{\d t} , g(t) \frac{\d}{\d t} , h(t) \frac{\d}{\d t} \bigg) = {\rm det} \begin{pmatrix} f & g & h \\ f' & g' & h' \\ f'' & g '' & h '' \end{pmatrix} (t=0) .
\eeqn
The section $\phi^0 = j^* \phi$ of $\clie^\bu(\jet(\cT))$ is
\beqn
\phi^0 \bigg(\alpha(z,\zbar) \frac{\partial}{\partial z} , \beta(z,\zbar) \frac{\partial}{\partial z} , \gamma(z ,\zbar) \frac{\partial}{\partial z} \bigg) = {\rm det} \begin{pmatrix} \alpha^0 & \beta^0 & \gamma^0 \\ \partial_z\alpha^0 & \partial_z\beta^0 & \partial_z \gamma^0  \\ \partial_z^2 \alpha^0 & \partial_z^2\beta^0 & \partial_z^2 \gamma^0\end{pmatrix} (z,\zbar) .
\eeqn
Here $\alpha^0$ denotes the zero form component of the differential form $\alpha$. 
We first solve for the descent element $\phi^{0,1}$ which satisfies the holomorphic descent equation
\beqn
\dbar \phi^0 = \dbar_\cT \phi^{0,1} .
\eeqn
This element has the form $\phi^{0,1} = \d \zbar \psi^{0,1}$ where $\psi^{0,1}$ is the section of $\clie^\bu(\jet (\cT))$ defined by
\beqn
\psi^{0,1} \bigg(\alpha \frac{\partial}{\partial z} , \beta \frac{\partial}{\partial z} , \gamma \frac{\partial}{\partial z} \bigg) = \phi^0\left(\alpha^{0,1} \frac{\partial}{\partial z} , \beta^0 \frac{\partial}{\partial z} , \gamma^0 \frac{\partial}{\partial z} \right) + \cdots
\eeqn
where $\cdots$ denotes the two terms obtained by swapping the role of $\alpha$ with $\beta,\gamma$ respectively. 
Next, we solve for $\phi^{1,1}$ which satisfies the Cartan descent equation
\beqn
\partial \phi^{0,1} = \d_{\rm CE} \phi^{1,1} .
\eeqn
Explicitly $\phi^{1,1} = \d z \d \zbar \psi^{1,1}$ with $\psi^{1,1}$ the section of $\clie^\bu(\jet (\cT))$ defined by
\beqn
\psi^{1,1} \bigg(\alpha \frac{\partial}{\partial z} , \beta \frac{\partial}{\partial z} \bigg) = \partial_z \alpha^{0,1} \partial_z^2 \beta^0  - \partial_z^2 \alpha^{0,1} \partial_z \beta^0 + \left(\alpha \leftrightarrow \beta \right) .
\eeqn
The local cocycle $\delta(\phi) = \int \phi^{1,1} \in \cloc^\bu(\cT(\CC))$ can be put in the following more uniform form
\beqn
\delta(\phi) = \int J(\xi) \wedge \partial J(\xi) .
\eeqn

\end{eg}

\appendix 

\section{Other Lie algebras of vector fields} \label{sec:variants}

\subsection{The result for subalgebras}

We briefly remark on similar results for other Lie algebras of holomorphic vector fields which can be 
realized as subalgebras of holomorphic vector fields.
The one we will pay most attention to is the Lie algebra of {\em symplectic vector fields}. 

Let $(X,\omega)$ be a holomorphic symplectic manifold of complex dimension $2n$. 
The sheaf of holomorphic symplectic vector fields, denoted $\cH^{\rm hol} (X)$, is the subsheaf of all holomorrphic vector fields
\beqn
\cH^{\rm hol} \hookrightarrow \cT^{\rm hol}
\eeqn
which preserve the symplectic form $\omega$. 

To formulate the notion of local cohomology we first need to start with a local Lie algebra.
Recall from \S \ref{sec:intro} that $\cT^{\rm hol}$ is not a local Lie algebra but its Dolbeault resolution $\cT$ is. 
Likewise, $\cH^{\rm hol}$ is not a local Lie algebra, but we can find a resolution that is. 
To define this resolution, first consider the following sheaf of cochain complexes $\Omega^{\geq 2, \bu}$ on $X$ defined by 
\beqn
\Omega^{2, \bu} \xto{\partial} \Omega^{3, \bu} [-1] \xto{\partial} \cdots \xto{\partial} \Omega^{2n, \bu}(X) [-2n+2] .
\eeqn
Here the $\dbar$ operator is left implicit and $\partial$ denotes the holomorphic de Rham operator. 
This sheaf is a free resolution of the sheaf of $\partial$-closed holomorphic two-forms on $X$. 

\begin{dfn}
Let $X$ be a holomorphic symplectic manifold.
Consider the complex of vector bundles on $X$:
\beqn
\Omega^{0,\bu}(X , {\rm T}) \xto{L_{(-)} \omega} \Omega^{\geq 2, \bu} (X) [-1]
\eeqn
and let $\cH$ denote its sheaf of sections. 
\end{dfn}

%Suppose $\cS \subset \cT$ is a local dg Lie subalgebra of $\cT$ such that for every open set $U \subset \CC^n$ the space $\cS(U)$ contains the vector fields $\left\{\frac{\partial}{\partial z_i}\right\}$. 
%Also, let $\fs$ be the Lie algebra $J_0 H^0(\cS)$, which is the $\dbar$-cohomology of the fiber of $J \cS$ over the point $0 \in \cS$. 
%
%\begin{thm}
%The map of cochain complexes
%\beqn
%\delta : \clie_{\rm red}^\bu(\fs) \xto{\simeq} \cloc^\bu(\cS(\CC^n)) [-2n] .
%\eeqn
%defined by $\delta(\phi) = \int \phi^{n,n}$, where $\phi^{n,n}$ is as in Equation (\ref{eqn:Phi}), is a quasi-isomorphism. 
%\end{thm}
%
%\begin{eg}
%An example of such a local dg Lie algebra $\cS$ is that of the divergence free vector fields on $\CC^n$ with respect to the standard holomorphic volume form $\d^n z$. 
%\end{eg}
%
%\begin{rmk}
%\brian{Hamiltonian vector fields also work.}
%\end{rmk}

\begin{prop}
Let $X$ be a holomorphic symplectic manifold of complex dimension $2n$ and let $\cT^{\rm symp}$ be the local Lie algebra of holomorphic symplectic vector fields.
Then, there is a quasi-isomorphism
\beqn
\cloc^\bu\left(\cT^{\rm symp} \right) \; \simeq \; \ul{\cred^\bu (\fh_{2n}}) [4n]  .
\eeqn
\end{prop}

\subsection{The smooth version}

\subsection{Super version}

\printbibliography

%\appendix
%
%\section{Smooth version}
%
%Let's first consider the smooth case. 
%Suppose $M$ is an $n$-dimensional manifold and let $\fX(M)$ denote the associated Lie algebra of vector fields. 
%For each $x \in M$ we have a cochain map
%\beqn
%j_x^* : \clie^\bu(\fw_n) \to \clie^\bu(\fX(M))
%\eeqn
%which sends a cochain $\alpha$ to $j_x^* \alpha$ where 
%\beqn
%j_x : \fX(M) \to \fw_n
%\eeqn
%takes a vector field and computes its $\infty$-jet at $x \in M$. 
%
%\begin{lem}
%Let $\alpha \in \clie^k(\fw_n)$ be a cochain. 
%For any $x,y \in M$ the cochains $j_x^* \alpha$ and $j_y^* \alpha$ are cohomologous. 
%In particular, there exists a $\alpha^{(1)} \in \Omega^1(M) \otimes \clie^{k-1}(\fw_n)$ such that $(\d_{dR} \otimes 1) \alpha = (1 \otimes \d_{\fX}) \alpha^{(1)}$. 
%\end{lem}
%
%Inductively, we obtain a sequence of cochains $(j_x^* \alpha, \alpha^{(1)}, \ldots, \alpha^{(n)})$ satisfying 
%\beqn
%(\d_{dR} \otimes 1) \alpha^{(j)} = (1 \otimes \d_{\fX}) \alpha^{(j+1)} .
%\eeqn
%It follows immediately that for any $\ell$-cycle $N \subset M$ one obtains a cochain of $\fX(M)$ with {\em trivial} coefficients:
%\beqn
%\int_N \alpha^{(\ell)} \in \clie^{k-\ell}(\fX(M)) .
%\eeqn
%
%\begin{lem}
%If $\alpha \in \clie^k(\fw_n)$ is a cocycle then $\int_N \alpha^{(\ell)}$ is a cocycle. 
%\end{lem}
%
%\begin{eg}
%Consider the cocycle $\alpha \in \clie^3(\fw_1)$ which is dual to the homology $3$-cycle
%\beqn
%L_{-1} \wedge L_0 \wedge L_1 \in {\rm C}_3(\fw_1) .
%\eeqn
%Consider $M = S^1$. 
%For $x \in S^1$ one has 
%\beqn
%j_x^* \alpha \bigg(f(x) \frac{\d}{\d x} , g(x) \frac{\d}{\d x} , h(x) \frac{\d}{\d x} \bigg) = \brian{you know the determinant} .
%\eeqn
%An explicit form of a cochain $\alpha^{(1)} \in \Omega^1(S^1) \otimes \clie^2(\fX(S^1))$ satisfying $\d_{\fX} \alpha^{(1)} = \d_{dR} j_x^* \alpha$ is
%\beqn
%\alpha^{(1)} \bigg( f(x) \frac{\d}{\d x} , g(x) \frac{\d}{\d x} \bigg) \define f'(x) \d_{dR} (g'(x)) - g'(x) \d_{dR} (f'(x)) \in \Omega^1(S^1) .
%\eeqn
%As one can immediately check, $\int_{S^1} \alpha^{(1)} \in \clie^2 (\fX(S^1))$ is the usual Virasoro cocycle.
%\end{eg}



\end{document}


