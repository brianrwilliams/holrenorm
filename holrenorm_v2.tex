\documentclass[10pt]{article}

\usepackage{macros}

%Formatting
\linespread{1.25}

\usepackage{parskip}
\setlength{\parindent}{18pt}
\setlength{\parindent}{0cm}

\setcounter{tocdepth}{2}
\setcounter{secnumdepth}{5}


\def\brian{\textcolor{blue}{BW: }\textcolor{blue}}

\title{Renormalization for holomorphic field theories}

\author{Brian R. Williams}

\begin{document}

\maketitle

\section{Introduction}

\section{The definition of a holomorphic field theory}

The goal of this section is to define the notion of a holomorphic field theory. 
This is a variant of Costello's definition of a BV theory, see the previous section, and we will take for granted that the reader is familiar with the general format.
In summary, we modify the definition of a theory by inserting the word ``holomorphic" in front of most objects (bundles, differential operators, etc..).
By applying the Dolbeault complex in appropriate locations, we will recover Costello's definition of a theory, but with a holomorphic flavor, see Table \ref{table: holtoBV}. 

There are many references in the physics literature to codify the concept of a holomorphic field theory.
See, most closely related to our approach, special cases of this in the work of Nekrasov and collaborators in \cite{NekThesis, NekChiral, NekCFT}. 
We will discuss in more detail the relationship of our analysis of holomorphic theories to this work in Chapters \ref{chap: holsig} and \ref{chap: symmetries}. 

\subsection{The definition of a holomorphic theory}

We give a general definition of a classical holomorphic theory on a general complex manifold $X$ of complex dimension $d$.
We start with the definition of a {\em free} holomorphic field theory. 
After that we will go on to define what an interacting holomorphic theory is.

\subsubsection{Free holomorphic theories}

The essential information that govern a classical field theory are its equations of motion. 
For a free theory, the equations of motion are linear in the space of fields.
Thus, at least classically, the setting of free theories can essentially be reduced to the study linear partial differential equations.
\brian{finish...}

In the formalism for field theory developed in \cite{CosRenorm} and \cite{CG1, CG2} the fields of a theory on a manifold $X$ are always expressed as sections of some $\ZZ$-graded vector bundle on $X$.
Here, the $\ZZ$-grading is the cohomological, or BRST \footnote{Named after Becchi, Rouet, Stora, Tyutin, for which our approach to field theory is greatly influenced by their original mathematical approach to quantization.}, grading of the theory.
For a {\em holomorphic theory} we will impose that this graded vector bundle be holomorphic.  
By a holomorphic $\ZZ$-graded vector bundle we mean a $\ZZ$-graded vector bundle $
V^\bullet = \oplus_i V^i [-i]$ (which we will usually abbreviate simply as $V$) such that each graded piece $V^i$ is a holomorphic vector bundle (here $V^i$ is in cohomological degree $+i$).
Thus, in order to define a holomorphic field theory on a complex manifold $X$ we start with the data:

\begin{itemize}
\item[(1)] a $\ZZ$-graded holomorphic vector bundle $V^\bullet = \oplus_i V^i [-i]$ on $X$, so that the finite dimensional holomorphic vector bundle $V^i$ is in cohomological degree $i$. 
\end{itemize}

\begin{rmk}
For supersymmetric theories it may be desirable to include an additional $\ZZ/2$, or fermionic, grading into the data of the space of fields, but we do not do that here.
\end{rmk}

A free classical theory is made up of a space of fields as above together with the data of a linearized BRST differential $Q^{BRST}$ and a shifted symplectic pairing of cohomological degree $-1$. 
Ordinarily, the BRST operator is simply a differential operator on the underlying vector bundle defining the fields. 
For the class of theories we are considering, we require this operator be holomorphic. 
For completeness, we briefly recall this notion.

Suppose that $E$ and $F$ are two holomorphic vector bundles on $X$.
Note that the Hom-bundle ${\rm Hom}(E,F)$ inherits a natural holomorphic structure. 
By definition, a {\em holomorphic differential operator of order $m$} is a linear map
\ben
D : \Gamma^{hol}(X ; E) \to \Gamma^{hol}(X ; F)
\een
such that, with respect to a holomorphic coordinate chart $\{z_i\}$ on $X$, $D$ can be written as
\be\label{local holomorphic}
D|_{\{z_i\}} = \sum_{|I| \leq m} a_I (z) \frac{\partial^{|I|}}{\partial z_I}
\ee
where $a_I(z)$ is a local holomorphic section of ${\rm Hom}(E,F)$.
Here, the sum is over all multi-indices $I = (i_1,\ldots, i_d)$ and 
\ben
\frac{\partial^{|I|}}{\partial z_I} := \prod_{k=1}^d \frac{\partial^{i_k}}{\partial z_k^{i_k}} . 
\een 
The length of the multi-index $I$ is defined by $|I| := i_1 + \cdots + i_d$. 

\begin{eg}
The most basic example of a holomorphic differential operator is the $\partial$ operator for the trivial vector bundle. 
For each $1 \leq \ell \leq d = \dim_\CC(X)$, it is a holomorphic differential operator from $E = \wedge^\ell T^{1,0*}X$ to $F = \wedge^{\ell+1} T^{1,0*}X$ which on sections is
\ben
\partial : \Omega^{\ell, hol}(X) \to \Omega^{\ell+1, hol}(X) .
\een
Locally, of course, it has the form
\ben
\partial = \sum_{i = 1}^{d} (\d z_i \wedge (-)) \frac{\partial}{\partial z_i},
\een
where $\d z_i \wedge (-)$ is the vector bundle homomorphism $\wedge^\ell T^{1,0*}X \to \wedge^{\ell+1} T^{1,0*}X$ sending $\alpha \mapsto \d z_i \wedge \alpha$. 
\end{eg}

The next piece of data we fix is:
\begin{itemize}
\item[(2)] a square-zero holomorphic differential operator 
\ben
Q^{hol} : \sV^{hol} \to \sV^{hol}
\een
of cohomological degree $+1$. 
Here $\sV^{hol}$ denotes the holomorphic sections of $V$. 
\end{itemize}

Finally, to define a free theory we need the data of a shifted symplectic pairing. 
For reasons to become clear in a moment, we must choose this pairing to have a strange cohomological degree. 
The last piece of data we fix is:
\begin{itemize}
\item[(3)] an invertible bundle map
\ben
(-,-)_V : V \tensor V \to K_X[d-1]
\een
Here, $K_X$ is the canonical bundle on $X$. 
\end{itemize}

The definition of the fields of an ordinary field theory are the {\em smooth} sections of the vector bundle $V$. 
In our situation this is a silly thing to do since we lose all of the data of the complex structure we used to define the objects above.
The more natural thing to do is to take the {\em holomorphic} sections of the vector bundle $V$. 
By construction, the operator $Q^{hol}$ and the pairing $(-,-)_V$ are defined on holomorphic sections, so on the surface this seems reasonable.
The technical caveat that the sheaf of holomorphic sections does not satisfy certain conditions necessary to study observables in our approach to QFT. 
For more details on this see Remark \ref{rmk: hol sec bad}.
We will take a natural resolution of holomorphic sections in order to relate to the usual definition of a classical BV theory.

Given any holomorphic vector bundle $V$ we can define its {\em Dolbeault complex} $\Omega^{0,*}(X , V)$ with its Dolbeault operator 
\ben
\dbar : \Omega^{0,p}(X, V) \to \Omega^{0,p+1}(X, V) .
\een
Here, $\Omega^{0,p}(X, V)$ denotes smooth sections of the vector bundle $\Wedge^p T^{0,1*} X \tensor V$. 
The fundamental property of the Dolbeault complex is that it provides a resolution for the sheaf of holomorphic sections $\sV^{hol} \simeq \Omega^{0,*}_X(V)$. 

We now take a graded holomorphic vector bundle $V$ as above, equipped with the differential operator $Q^{hol}$. 
We can then define the totalization of the Dolbeault complex with the operator $Q^{hol}$:
\ben
\sE_V = \left(\Omega^{0,*}(X, E), \dbar + Q^{hol}\right) .
\een
The operator $\dbar + Q^{hol}$ will be the linearized BRST operator of our theory.
By assumption, we have $\dbar Q^{hol} = Q^{hol} \dbar$ so that $(\dbar + Q^{hol})^2 = 0$ and hence the fields still define a complex. 
The $(-1)$-shifted symplectic pairing is obtained by composition of the pairing $(-,-)_V$ with integration on $\Omega^{d,hol}_X$. 
The thing to observe here is that $(-,-)_V$ extends to the Dolbeault complex in a natural way: we simply combine the wedge product of forms with the pairing on $V$.
The $(-1)$-shifted pairing $\omega_V$ on $\sE$ is defined by the diagram
\ben
\xymatrix{
\sE_V \tensor \sE_V \ar[r]^-{(-,-)_V} \ar@{.>}[dr]_-{\omega_V} & \Omega^{0,*}(X , K_X) [d-1] \ar[d]^-{\int_X} \\
& \CC[-1] .
}
\een
We note that the top Dolbeault forms with values in the canonical bundle $K_X$ are precisely the top forms on the smooth manifold $X$, so integration makes sense. 

We arrive at the following definition. 

\begin{dfn/lem}\label{dfn hol free theory}
A {\em free holomorphic theory} on a complex manifold $X$ is the data $(V, Q^{hol}, (-,-)_V)$ as in (1), (2), (3) above such that $Q^{hol}$ is a square zero elliptic differential operator that is graded skew self-adjoint for the pairing $(-,-)_V$.
The triple $(\sE_V, Q_V = \dbar + Q^{hol}, \omega_V)$ defines a free BV theory in the usual sense.
\end{dfn/lem}

The usual prescription for writing down the associated action functional holds in this case.
If $\varphi \in \Omega^{0,*}(X , V)$ denotes a field the action is
\ben
S(\varphi) = \int_X \left(\varphi, (\dbar + Q^{hol}) \varphi \right)_V .
\een

The first example we explain is related to the subject of Chapter \ref{chap: holsig} and will serve as the fundamental example of a holomorphic theory. 

\begin{eg}\label{eg bg} {\em The free $\beta\gamma$ system}.
Suppose that 
\ben
V = \ul{\CC} \oplus K_X [d - 1] .
\een
Let $(-,-)_V$ be the pairing
\ben
(\ul{\CC} \oplus K_X) \tensor (\ul{\CC} \oplus K_X) \to K_X \oplus K_X \to K_X 
\een 
sending $(\lambda, \mu) \tensor (\lambda',\mu') \mapsto (\lambda \mu', \lambda'\mu) \mapsto \lambda\mu' + \lambda' \mu$.
In this example we set $Q^{hol} = 0$. 
One immediately checks that this is a holomorphic free theory as above.
The space of fields can be written as
\ben
\sE_V = \Omega^{0,*}(X) \oplus \Omega^{d,*}(X)[d - 1] .
\een 
We write $\gamma \in \Omega^{0,*}(X)$ for a field in the first component, and $\beta \in \Omega^{d,*}(X)[d - 1]$ for a field in the second component. 
The action functional reads
\ben
S(\gamma + \beta, \gamma'+\beta') = \int_{X} \beta \wedge \dbar \gamma' + \beta' \wedge \dbar \gamma .
\een 
When $d = 1$ this reduces to the ordinary chiral $\beta\gamma$ system from conformal field theory. 
The $\beta\gamma$ system is a bosonic version of the ghost $bc$ system that appears in the quantization of the bosonic string, see Chapter 6 of \cite{Polchinski1}.
We will study this higher dimensional version further in Chapter \ref{chap: holsig}.
For instance, we will see how this theory is the starting block for constructing general holomorphic $\sigma$-models. 
\end{eg}

Of course, there are many variants of the $\beta\gamma$ system that we can consider.
For instance, if $E$ is {\em any} holomorphic vector bundle on $X$ we can take 
\ben
V = E \oplus K_{\CC^d} \tensor E^\vee
\een
where $E^\vee$ is the linear dual bundle. 
The pairing is constructed as in the case above where we also use the evaluation pairing between $E$ and $E^\vee$.
In thise case, the fields are $\gamma \in \Omega^{0,*}(X, E)$ and $\beta \in \Omega^{d,*}(X, E^\vee)[d-1]$. 
The action functional is simply
\ben
S(\gamma, \beta) = \int {\rm ev}_E(\beta \wedge \dbar \gamma) .
\een
When $E$ is a tensor bundle of type $(r,s)$ this theory is a bosonic version of the $bc$ ghost system of spin $(r,s)$. 
For a general bundle $E$ we will refer to it as the $\beta\gamma$ system with coefficients in the bundle $E$. 

%\begin{eg}
%{\em The free chiral scalar}.
%Another basic example is the free chiral scalar. 
%This is a bit outside\brian{finish}
%Let $X$ be a complex manifold with Hermitian metric $g$. 
%Let $V = \ul{\CC}$, the trivial vector bundle. 
%\brian{do this}
%\end{eg}

\begin{rmk} \label{rmk: hol sec bad}
We will only work with a holomorphic theory prescribed by the data $(V, (-,-)_V, Q^{hol})$ through its associated BV theory.
One might propose a definition of a BV theory in the analytic category based off of holomorphic sections of holomorphic vector bundles. 
There are numerous technical reason why this approach fails in our approach to QFT.
In particular, the sheaf of holomorphic sections of a holomorphic bundle is not fine, and there do not exists partitions of unity in general. 
\brian{why is this bad?}
\end{rmk}

\subsubsection{Interacting holomorphic theories} \label{sec: interacting}

\def\olochol{\sO_{\rm loc}^{hol}}

We proceed to define what an interacting holomorphic theory is.
Roughly, a general interacting field theory with space of fields $\sE$ is prescribed by a functional
\ben
S : \sE \to \CC
\een
that satisfies the {\em classical master equation}.
The key technical condition is that this functional must, in addition, be {\em local}.
We proceed to briefly recall this notion.

First off, consider the algebra of functions $\sO(\sE(X))$ on the space of sections $\sE(X)$.
Similarly, let $\sO_{red}(\sE(X)) = \sO(\sE(X)) / \RR$ be the quotient by the constant polynomial functions. 
For recollections on our conventions for these spaces of functions, see Appendix \brian{ref}.

If $E$ is any graded vector bundle on $X$ let ${\rm Jet}(E)$ denote its bundle of $\infty$-jets. 
This is a smooth vector bundle, albeit infinite rank, on $X$ whose fiber over $w \in X$ can be identified with
\ben
E_w \times \CC[[z_1,\ldots,z_d, \Bar{z}_1,\ldots,\Bar{z}_d]] .
\een
This object can be given the natural structure of a pro object in the category of vector bundles.
See, for example, Section \brian{...}.
We let $J(E)$ denotes the associated sheaf of smooth sections.
It is well-known that ${\rm Jet}(E)$ is equipped with a natural flat connection rendering $J(E)$ with the structure of a smooth $D_X$-module.

The space $\sO_{red}(J(E))$ inherits a natural $D_X$-module structure from $J(E)$. 
We refer to $\sO_{red}(J(E))$ as the space of {\em Lagrangians} on the vector bundle $E$. 
Every element $F \in \sO_{red}(J(E))$ can be expanded as $F = \sum_n F_n$ where each $F_n$ is an element 
\ben
F_n \in {\rm Hom}_{C^\infty_X} (J(E)^{\tensor n}, C^\infty_X)_{S_n} \cong {\rm PolyDiff}(\sE^{\tensor n}, C^\infty(X))_{S_n}
\een
where the right-hand side is the space of polydifferential operators.
The proof of the isomorphism on the right-hand side can be found in Chapter 5 of \cite{CostelloRenormalization}.

A local functional is prescribed by the data of a Lagrangian modulo terms involving total derivatives.
The mathematical definition is the following.

\begin{dfn} \label{dfn: local fnl}
Let $E$ be a graded vector bundle on $X$.
Define the sheaf of {\em local functionals} on $X$ to be
\ben
\oloc(\sE) = {\rm Dens}_X \tensor_{D_X} \sO_{red}(J(E)),
\een
where we use the natural right $D_X$-module structure on densities.
\end{dfn}

%Recall, the sheaf of local functionals on $\sE = \Gamma(E)$ is defined as the sheaf of Lagrangian densities
%\ben
%\oloc(\sE) = {\rm Dens}_M \tensor_{D_M} \sO_{red}(JE) .
%\een
%In the expression above $JE$ stands for the sheaf of smooth sections of the $\infty$-jet bundle ${\rm Jet}(E)$ which has the structure of a $D_X$-module.

If $V$ is a holomorphic vector bundle we can define the bundle of holomorphic $\infty$-jets ${\rm Jet}^{hol}(V)$, \cite{GriffithsGreen, WongChandler}. 
This is a pro-vector bundle that is holomorphic in a natural way.
The fibers of this infinite rank bundle ${\rm Jet}^{hol}(V)$ are isomorphic to 
\ben
{\rm Jet}^{hol}(V)|_w = V_w \tensor \CC[[z_1,\ldots,z_d]],
\een
where $w \in X$ and where $\{z_i\}$ is the choice of a holomorphic formal coordinate near $w$. 
We denote by $J^{hol} V$ the sheaf of holomorphic sections of this jet bundle.
The sheaf $J^{hol}V$ has the structure of a $D_X^{hol}$-module, that is, it is equipped with a holomorphic flat connection $\nabla^{hol}$.
This situation is completely analogous to the smooth case.
Locally, the holomorphic flat connection on ${\rm Jet}^{hol}(V)$ is of the form
\ben
\nabla^{hol} |_w = \sum_{i=1}^d \d w_i \left(\frac{\partial}{\partial w_i} - \frac{\partial}{\partial z_i}\right),
\een
where $\{w_i\}$ is the local coordinate on $X$ near $w$ and $z_i$ is the fiber coordinate labeling the holomorphic jet expansion.
%Using holomorphic jets we can make a completely analogous definition in our setting.

One natural appearance of the bundle of holomorphic jets is in providing an explicit description of holomorphic differential operators. 
Indeed, holomorphic differential operators are the same as bundle maps between the associated holomorphic jet bundles. 
A similar result holds for {\em poly}differential operators, which we now recall.
Suppose $V,W$ are holomorphic vector bundles with spaces of holomorphic sections given by $\sV^{hol},\sW^{hol}$ respectively.
Then we can express $n$-ary polydifferential operators from $V$ to $W$ as
\ben
{\rm PolyDiff}^{hol}(\underbrace{\sV^{hol} \times \cdots \times \sV^{hol}}_{n \; {\rm copies}}, \sW^{hol}) \cong {\rm Hom}(\underbrace{{\rm Jet}^{\rm hol}(V) \tensor \ldots \tensor {\rm Jet}^{\rm hol}(V)}_{n \; {\rm copies}} , W) .
\een
The right-hand side denotes hom's in the category of holomorphic vector bundles.

\begin{dfn}\label{dfn hol lag}
Let $V$ be a vector bundle. 
The space of {\em holomorphic Lagrangian densities} on $V$ is
\ben
\sO_{red}^{hol}(V) = \prod_{n > 0} {\rm Hom} ({\rm Jet}^{hol}(V)^{\tensor n} , K_X)_{S_n} ,
\een
where Hom is taken in the category of holomorphic vector bundles.\footnote{The holomorphic vector bundle ${\rm Jet}^{hol}(V)$ is infinite dimensional and can be expressed as a pro-object in the category of holomorphic vector bundles. 
We require the bundle maps to be continuous with respect to the natural adic topology.}
Equivalently, a holomorphic Lagrangian density is of the form $F = \sum_n F_n \in \sO_{red}^{hol}(V)$ where each $F_n$ is a holomorphic polydifferential operator 
\ben
F_n : \sV^{hol} \times \cdots \times \sV^{hol} \to \Omega^{d,hol}_X .
\een
\end{dfn}

%\begin{dfn}
%Suppose $V$ is a graded holomorphic vector bundle.
%We define the sheaf of {\em holomorphic} local functionals on $V$ by
%\ben
%\olochol(V) = \Omega^{d,hol}_X \tensor_{D^{hol}_X} \sO_{red}(J^{hol}V) [d]
%\een
%\end{dfn}

Next, suppose that $V$ is part of the data of a free holomorphic theory $(V, Q^{hol},(-,-)_V)$.
The pairing $(-,-)_V$ endows the space of holomorphic Lagrangians with a sort of bracket.
Indeed, suppose $F, F' \in \sO_{red}^{hol}(V)$.
For simplicity we assume $F,F'$ are of homogenous symmetric degree $k,k'$ respectively.
Then, their product $F \tensor F'$ is an element in the homomorphism space
\ben
{\rm Hom}({\rm Jet}^{hol}(V)^{\tensor (k +k')} , K_X \tensor K_X) .
\een
The bundle map $(-,-)_V : V \tensor V \to K_X[d-1]$ is invertible, hence it determines an element $(-,-)^{-1}_V \in V \tensor V \tensor K_X^\vee$ of cohomological degree $d-1$, where $K_X^\vee$ is the dual bundle. 
We can then consider the composition
\ben
\xymatrix{
{\rm Hom}({\rm Jet}^{hol}(V)^{\tensor (k + k')} , K_X \tensor K_X) \ar[r]^-{(-,-)^{-1}_V} & {\rm Hom}({\rm Jet}^{hol}(V)^{\tensor (k +k' -2)} , K_X^\vee \tensor K_X \tensor K_X) \ar[r] & {\rm Hom}({\rm Jet}^{hol}(V)^{\tensor (k +k' -2)} , K_X).
}
\een
In the first arrow we have evaluated $(-,-)^{-1}_V$ on the first two factors and the second arrow is simply the evaluation pairing. 
We symmetrize this to obtain an element $\{F, F'\}^{hol} \in {\rm Hom}({\rm Jet}^{hol}(V)^{\tensor (k+k'-2)} , K_X)_{S_{k+k'-2}}$. 
Extending to inhomogenous polynomial degree functionals is done in the obvious way.
Thus, we have produced a bilinear map
\ben
\{-,-\}^{hol} : \sO_{red}^{hol}(V) \times \sO_{red}^{hol}(V) \to \sO_{red}^{hol}(V) [d-1].
\een
Note that this bracket is of cohomological degree $-d+1$ and lowers the polynomial degree by two.

%\begin{rmk}
%Note that the above still makes sense if $(-,-)_V$ is a polydifferential operator...\brian{should i say more?}
%\end{rmk}

We can now state the definition of a classical holomorphic theory. 

\begin{dfn}
A {\em classical holomorphic theory} on a complex manifold $X$ is the data of a free holomorphic theory $(V, Q^{hol}, (-,-)_V)$ plus a functional
\ben
I^{hol} \in \prod_{n \geq 3} {\rm Hom} ({\rm Jet}^{hol}(V)^{\tensor n} , K_X)_{S_n} \subset \sO_{red}^{hol}(V) 
\een
of cohomological degree $d$ such that the following expression
\ben
Q^{hol} I^{hol} + \frac{1}{2} \{I^{hol}, I^{hol}\}^{hol} 
\een 
is $\partial$-exact.
\end{dfn} 

\begin{rmk}
Note that in the definition we require that the functional $I^{hol}$ be at least cubic.
For brevity, we will denote the subspace $\prod_{n \geq 3} {\rm Hom} ({\rm Jet}^{hol}(V)^{\tensor n} , K_X)_{S_n} \subset \sO_{red}^{hol}(V)$ by $\sO_{red}^{hol}(V)^{+}$. 
\end{rmk}

%\begin{dfn/lem}
%Let $(V, Q^{hol}, (-,-)_V, I^{hol})$ be the data of an interacting holomorphic theory. 
%Then $Q^{hol} + \{I^{hol},-\}$ equips $\olochol(V)$ with the structure of a sheaf of cochain complexes that we will denote
%\ben
%\Def^{hol}_{V} := \left(\olochol(V), Q^{hol} + \{I^{hol}, -\}^{hol}\right) .
%\een
%\end{dfn/lem}

As in the free case, we proceed to verify that a holomorphic theory defines an interacting classical BV theory in the sense of Definition \brian{ref??} in \cite{CosRenorm, CG2}. 

The underlying space of fields, as we have already seen in the free case, is $\sE_V = \Omega^{0,*}(X , V)$. 
We will write $I^{hol} = \sum_k I^{hol}_k$ where $I^{hol}_k$ is symmetric degree $k$.
Now, we know that $I^{hol}_k$ is a $\Omega^{d,hol}_X$-valued functional of the form
\ben
I_k^{hol} : (\varphi_1,\ldots,\varphi_k) \mapsto D_1(\varphi_1)\cdots D_k(\varphi_k) \in \Omega^{d,hol}_X
\een
where $\varphi_i\in \sV = \Gamma^{hol}(X, V)$ and $D_i$ is a holomorphic differential operator $\sV \to \Omega^{l, hol} \tensor \sV$ for some $l$.

Every holomorphic differential operator $D$ of the above form extends to a differential operator on the associated Dolbeault complexes.
Indeed, suppose $D : \sV \to \Omega^{l,hol} \tensor \sV$ is locally of the form $D = \omega \frac{\partial^{k_1}}{\partial z_1^{k_1}} \cdots \frac{\partial^{k_d}}{\partial z^{k_d}}$, where $\omega \in \Omega^{l,hol}$.
Then, if $\alpha = s_I (z,\zbar) \d \zbar_I \in \Omega^{0,*}(X, V)$, where $s_I$ is a local section of $T^*X \tensor V$, we define
\ben
D \alpha = \omega \wedge \left(\frac{\partial^{k_1}}{\partial z_1^{k_1}} \cdots \frac{\partial^{k_d}}{\partial z^{k_d}} s_I(z,\zbar)\right) \d \zbar_I \in \Omega^{l,*}(X , V) .
\een
In this way, $D$ extends to a differential operator 
\ben
D^{\Omega^{0,*}} : \Omega^{0,*}(X, V) \to \Omega^{l,*}(X,V) . 
\een

Via this construction, we can extend $I^{hol}$ to a $\Omega^{d,*}(X)$-valued functional on $\Omega^{0,*}(X, V)$ by the formula
\ben
I^{\Omega^{0,*}}_k : (\alpha_1,\ldots,\alpha_k) \mapsto D_1(\alpha_1) \cdots D_k(\alpha_k) \in  \Omega^{d,*}(X) .
\een

Furthermore, via integration we can define the $\CC$-valued functional on compactly supported sections
\ben
I_k = \int_X I^{0,*}_k : \Omega_c^{0,*}(X, V)^{\tensor k} \to \CC .
\een
Performing this for each homogenous piece, we obtain the functional $I = \sum_k \int I^{\Omega^{0,*}}_k$. 

The symbol $\int_X$ reminds us that we are working modulo total derivatives.
Moreover, since we began with a functional involving differential operators, we see that the above expression defines an element of $\oloc(\sE_V)$. 
In conclusion, our construction thus outlined determines a linear map $\olochol(V) \to \oloc(\sE_V)$ obtained via the composition $I^{hol} \mapsto I^{\Omega^{0,*}} \mapsto I = \int I^{\Omega^{0,*}}$. 
Note that since $I^{hol}$ is cohomological degree $d$, the local functional $I^{\Omega^{0,*}}$ is degree zero since integration is degree $-d$. 

\begin{lem} Every classical holomorphic theory $(V, Q^{hol},(-,-)_V, I^{hol})$ determines the structure of a classical BV theory.
The underlying free BV theory is given in Definition/Lemma \ref{dfn hol free theory} $(\sE_V, Q, \omega_V)$ and the interaction is $I = \int I^{\Omega^{0,*}}$. 
\end{lem}

\begin{proof}
We must show that $Q^{hol}I^{hol} + \frac{1}{2} \{I^{hol},I^{hol}\}^{hol}$ exact implies the ordinary classical master equation for $I$:
\ben
\dbar I + Q^{hol}I + \frac{1}{2} \{I,I\} = 0 .
\een
Since $I^{\Omega^{0,*}}$ is defined using holomorphic differential operators, we see that $(\dbar I^{\Omega^{0,*}} ) (\alpha) = \dbar (I^{\Omega^{0,*}} (\alpha))$. 
Thus, upon integration, we see that the first term is identically zero.

Now, since $Q^{hol}I^{hol} + \frac{1}{2} \{I^{hol},I^{hol}\}^{hol}$ is $\partial$-exact, $Q^{hol}I^{\Omega^{0,*}} + \frac{1}{2} \{I^{\Omega^{0,*}},I^{\Omega^{0,*}}\}$ is also $\partial$-exact.
By integration we obtain the result.
\end{proof}

Table \ref{table: holtoBV} is a useful summary showing how we are producing a BV theory from a holomorphic theory.

\begin{table}
\begin{center}
\begin{tabular}{ |c|c|c| } 
 \hline
 Holomorphic theory & BV theory \\
 \hline \hline
Holomorphic bundle $V$ & Space of fields $\sE_V = \Omega^{0,*}(X, V)$  \\ 
Holomorphic differential operator $Q^{hol}$ & Linear BRST operator $\dbar + Q^{hol}$ \\ 
Non-degenerate pairing $(-,-)_V$ & $(-1)$-symplectic structure $\omega_{V}$ \\ 
Holomorphic Lagrangian $I^{hol}$ & Local functional $I = \int I^{\Omega^{0,*}} \in \oloc(\sE_V)$ \\ 
 \hline
\end{tabular}
\caption{From holomorphic to BV}
\label{table: holtoBV}
\end{center}
\end{table}



\begin{eg} {\em Holomorphic $BF$-theory}
Let $\fg$ be a Lie algebra and $X$ any complex manifold.
Consider the following holomorphic vector bundle on $X$:
\ben
V = \ul{\fg}_X [1] \oplus K_X \tensor \fg^\vee [d-2] .
\een
The notation $\ul{\fg}_X$ denotes the trivial bundle with fiber $\fg$. 
The pairing $V \tensor V \to K_X[d-1]$ is similar to the pairing for the $\beta\gamma$ system, except we use the evaluation pairing $\<-.-\>_\fg$ between $\fg$ and its dual $\fg^\vee$. 
In this example, $Q^{hol} = 0$.

We describe the holomorphic Lagrangian.
If $f_i : X \to \CC, i=1,2$ are holomorphic functions and $\beta \in K_X$, consider the trilinear functional
\ben
I^{hol} (f_1 \tensor X_1, f_2 \tensor X_2, \beta \tensor X^\vee) = f_1f_2 \beta \<X^\vee, [X_1,X_2]\>_\fg + \cdots
\een
where the $\cdots$ means that we symmetrize the inputs.
This defines an element $I^{hol} \in \olochol(V)^+$ and the Jacobi identity for $\fg$ guarantees $\{I^{hol}, I^{hol}\}^{hol} = 0$. 
The fields of the corresponding BV theory are
\ben
\sE_V = \Omega^{0,*}(X, \fg)[1] \oplus \Omega^{d,*}(X, \fg^*) [d-2] .
\een
The induced local functional $I^{\Omega^{0,*}}$ on $\sE_V$ is
\ben
I^{\Omega^{0,*}} (\alpha, \beta) = \int_X \<\beta, [\alpha,\alpha]\>_\fg .
\een
The total action is $S(\alpha,\beta) = \int \<\beta, \dbar \alpha\> + \<\beta,[\alpha,\alpha]\>_\fg$.
This is formally similar to $BF$ theory (see below) and for that reason we refer to it as {\em holomorphic} BF theory. 
The moduli problem this describes is the cotangent theory to the moduli space of holomorphic connections on the trivial $G$-bundle near the trivial bundle.
There is an obvious enhancement that works near any holomorphic principal bundle.
When $d = 2$, in \cite{johansen1}, or for a more mathematical treatment see \cite{CostelloYangian}, it is shown that this theory is a twist of $N=1$ supersymmetric Yang-Mills on $\RR^4$.
\end{eg}

\begin{eg} {\em Topological $BF$-theory}
This is a deformation of the previous example that has appeared throughout the physics literature.
Suppose we take as our graded holomorphic vector bundle 
\ben
V = \left(\ul{\fg}_X \tensor \left(\oplus_{k = 0}^d \wedge^k T^{*1,0}X [1-k]\right)\right) \oplus \left(\ul{\fg^*}_X \tensor \left(\oplus_{k = 0}^d \wedge^k T^{*1,0}X [2(d-1)-k]\right)\right) .
\een
Here $\wedge^0 T^{*1,0}X$ is understood as the trivial bundle $\ul{\CC}_X$. 
The pairing is given by combining the evaluation pairing between $\fg$ and $\fg^*$ and taking the wedge product and projecting onto the components isomorphic to $K_X$.
Explicitly, the pairing is equal to the sum of bundle maps of the form
\ben
\ev_{\fg} \tensor \wedge : \left(\ul{\fg}_X \tensor \wedge^k T^{*1,0}X [1-k]\right) \tensor \left(\ul{\fg^*}_X \tensor \wedge^{d-k} T^{*1,0}X [d-1+k]\right) \to K_X [d-1] .
\een
The differential is of the form 
\ben
Q^{hol} = {\rm id}_\fg \tensor \partial + {\rm id}_{\fg^*} \tensor \partial,
\een
where $\partial$ is the holomorphic de Rham differential.
The holomorphic interaction is given by combining the Lie algebra structure on $\fg$ with the wedge product of the holomorphic bundles $\wedge^k T^{*1,0}X$. 
We observe that the associated BV theory has classical space of fields given by
\ben
(A,B) \in \sE_V = \Omega^*(X, \fg[1] \oplus \fg^*[2d-2]) 
\een
where $\Omega^*$ is now the {\em full} de Rham complex.
The action functional is
\ben
S = \int_X \<B, \d A\>_\fg + \frac{1}{3} \<B, [A,A]\>_\fg .
\een
As above, $\<-,-\>_\fg$ denotes the pairing between $\fg$ and its dual.
This is the well-known topological BF theory on the even dimensional {\em real} manifold $X$ (of real dimension $2d$). 
It might seem silly that we have used the formalism of holomorphic field theory to describe a very simple topological theory.
We will discuss advantages of this approach at the send of the next section.
In particular, the theory of regularization for holomorphic theories we will employ has peculiar consequences for renormalizing certain classes of topological theories such as Topological BF theory.
\end{eg}

When we construct a BV theory from a holomorphic theory $V \rightsquigarrow \sE_V$ it is natural to expect that deformations of the theory must come from holomorphic data.
In the special case that $Q^{hol} = 0$ we have the following result which relates the deformation complex of the classical theory $\sE_V$ to a sheaf built from holomorphic differential operators.

\begin{lem}
Suppose $(V, 0, (-,-)_V, I^{hol})$ is the data of a holomorphic theory with $Q^{hol} = 0$.
Let $(\sE_V, Q = \dbar, \omega_V, I)$ be the corresponding BV theory.
Then, there is a quasi-isomorphism of sheaves
\ben
\Def_{\sE_V}  \simeq \Omega^{d,hol}_X \tensor^{\LL}_{D_X^{hol}} \sO_{red}(J^{hol}V) [d]
\een
that is compatible with the brackets and $\{-,-\}$ and $\{-,-\}^{hol}$ on both sides.
\end{lem}
\begin{proof}
By definition, the deformation complex is equal to 
\ben
\Def_{\sE_V} = {\rm Dens}_X \tensor_{D_X} \sO_{red}(J \sE_V) .
\een
Since $\sO_{red}(J \sE_V)$ is flat as a $D_X$-module \cite{CosRenorm}, we can replace the tensor product $\tensor_{D_X}$ with the derived tensor product $\tensor^{\LL}_{D_X}$.

The remainder of the proof uses the following observation about $D$-modules.
If $M$ is a holomorphic $D_{X}^{hol}$-module, then clearly it forgets down to an ordinary smooth $D_X$-module (with the same underlying $C^\infty_X$-module structure) that we denote $M^{C^\infty}$. 
Moreover, there is a quasi-isomorphism of $D$-modules
\ben
\Omega^{d,d}_{X} \tensor^{\LL}_{D_X} M^{C^\infty} \simeq \Omega^{d,hol}_X \tensor^{\LL}_{D_X^{hol}} M [d] .
\een
To prove this we use the following ``anti-holomorphic" Spenser resolution. 
First, notice that $\Omega^{d,d}_X = \Omega^{d,0}_X \tensor_{C^\infty_X} \Omega^{0,d}_X$.
Next, consider the free resolution of $(0,d)$-forms as a...\brian{finish}

We apply this to the case $M = \sO_{red}(J^{hol} V)$, where $V$ is a holomorphic vector bundle.
To complete the proof, we need to show that for any holomorphic vector bundle $V$ that there is a quasi-isomorphism of $D_X$-modules between $\sO_{red}(J \Omega^{0,*}(X , V))$ and $\sO_{red}(J^{hol}V)$. 
For this, it suffices to show that the space of linear functionals are quasi-isomorphic. 
For any vector bundle $E$ there always exists a (non-canonical) splitting $J E \cong \sE \tensor_{C^\infty_X} J_X$, where $\sE$ is the sheaf of sections and $J_X$ is the sheaf of $\infty$-jets of the trivial bundle.
Thus, we can assume that $V$ is the trivial vector bundle, where the claim is now $(J\Omega^{0,*}(X))^\vee \simeq (J^{hol}_X)^\vee$. 
Both sides are quasi-isomorphic to the smooth sections of the bundle of holomorphic differential operators $D^{hol}$, so we are done. 
\end{proof}

\begin{rmk}
There is an alternative formulation of classical field theory in terms of sheaves of $L_\infty$ algebras, see Chapter \brian{ref} in \cite{CG2}.
Just as in the ordinary case we can formulate the data of a classical holomorphic theory in terms of sheaves of $L_\infty$ algebras. 
We will not do that here, but hope the idea of how to do so is clear.
\end{rmk}

\begin{rmk}
\brian{relate to Si Li's, Dijkraaf notion of chiral theory.}
\end{rmk}


\section{One-loop regularization for theories on $\CC^d$} \label{sec: hol renorm}

In this section we study the renormalization of holomorphically field theories on $\CC^d$ for general $d \geq 1$. 
We start with a classical holomorphic theory on $\CC^d$ and consider its one-loop homotopy renormalization group flow from some finite scale $\epsilon$ to scale $L$.
Explicitly, this flow manifests as a sum over weights of graphs; that is, {\em Feynman diagrams}.
In terms of diagrams we consider the sum over graphs of genus at most one where at each vertex we place the holomorphic interaction defining the classical theory.
To obtain a quantization of a classical theory one must make sense of the $\epsilon \to 0$ limit of this construction. 
In general, this involves introducing a family of {\em counterterms}.
The presence of counterterms can be a \brian{ref}. 
For instance, some theories of gravity require the introduction of infinitely many such counterterms.
Our main result is that for a holomorphic theory on $\CC^d$ no counterterms are required, and one obtains a well-defined $\epsilon \to 0$ limit. 

The model for holomorphic theories on $\CC^d$ we use follows our setup in the previous section.
Suppose $(V, Q^{hol}, (-,-)_V)$ prescribes the data of a free holomorphic theory on $\CC^d$.
This means that $V$ is a holomorphic bundle on $\CC^d$, $Q^{hol} : \sV^{hol} \to \sV^{hol}$ is a holomorphic differential operator, and $(-,-)_V$ is a (shifted) $K_{\CC^d}$-valued pairing on $V$. 
We assume, in addition, that $Q^{hol}$ is translation invariant.
Concretely, this means that
\ben
Q^{hol} \in \CC \left[\frac{\partial}{\partial z_1} , \ldots, \frac{\partial}{\partial z_d}\right].
\een

We can write the fields, in the BV formalism, as the following deformed Dolbeault complex
\ben
\sE_V = \left(\Omega^{0,*}(\CC^d, V), \dbar + Q^{hol}\right) .
\een
We will fix a trivialization for the holomorphic vector bundle $V = \CC^d \times V_0$, where $V_0$ is the fiber over $0 \in \CC^d$.
This leads to an identification $\Omega^{0,*}(\CC^d , V) = \Omega^{0,*}(\CC^d) \tensor_\CC V_0$.
Further, we write the $(-1)$-shifted symplectic structure defining the classical BV theory in the form
\ben
\omega_V(\alpha \tensor v, \beta \tensor w) = (v,w)_{V_0} \int \d^d z (\alpha \wedge \beta)
\een
where $(-,-)_{V_0}$ is a degree $(d-1)$-shifted pairing on the finite dimensional vector space $V_0$. 

A holomorphic interacting theory is prescribed by a holomorphic Lagrangian $I^{hol} \in \olochol(V)^+$. 
As we have seen in Section \ref{sec: interacting} any holomorphic Lagrangian determines a local functional on its Dolbeualt complex via integration $I = \int_X I^{\Omega^{0,*}}$. 
Here, as above, the notation $I^{\Omega^{0,*}}$ denotes the canonical extension of $I^{hol}$ to the Dolbeualt complex for $V$. 
Using the trivialization $V = \CC^d \times V_0$ and $\Omega^{d,hol} = \CC \cdot \d^d z$, we can express the local functional as
\ben
I_k (\alpha) = \int I^{hol}_k(\alpha) = \int D_{k,1}(\phi_{k,1} (\alpha))\cdots D_{k,k} ( \phi_{k,k}(\alpha)) \d^d z 
\een
where each $D_{i,j}$ is a holomorphic differential operator $D_{i,j} \in \CC\left[\frac{\partial}{\partial z_i}\right]$, and $\phi_{i,j} \in V_0^\vee$.

\subsection{Homotopy RG flow}

As we've already mentioned, the main goal of this section is to show that for holomorphic theories on $\CC^d$ the one-loop renormalization group flow produces a prequantization modulo $\hbar^2$. 
Recall, a prequantization is a an effective family of functionals satisfying renormalization group flow but not necessarily the quantum master equation. We study the consequences for solving the quantum master equation modulo $\hbar^2$ in the next section.  

With the requisite notation set up in the previous section, we move towards the main calculation, which will amount to producing an explicit bound on certain one-loop Feynman diagrams. 
Before proceeding with the core analysis, we recall the definition of homotopy renormalization group flow, as defined in \cite{CosRenorm}, which used to define the prequantization.

The building block in Costello's approach to renormalization is an effective family of functionals $\{I[L]\}$ parametrized by a {\em length scale} $L > 0$. 
For each $L > 0$ the functional $I[L] \in \sO(\sE)[[\hbar]]$ must satisfy various conditions. 
The first of which is a compatibility between the functionals as one changes the length scale; this is referred to as {\em homotopy renormalization group (RG) flow}.
The flow from scale $L$ to $L'$ is encoded by a linear map
\ben
W(P_{L < L'} , -) : \sO(\sE)[[\hbar]] \to \sO(\sE)[[\hbar]]
\een
defined as a sum over weights of graphs $W (P_{L<L'}, I) = \sum_{\Gamma} W_{\Gamma}(P_{L<L'}, I)$. 
Here, $\Gamma$ denotes a graph \footnote{For our purposes, a graph...\brian{finish}}, and the weight $W_\Gamma$ is defined as follows.

\brian{finish this}

The family of functionals $\{I[L]\}$ defining a quantization must satisfy the {\em RG flow equation}
\ben
I[L'] = W(P_{L<L'}, I[L])
\een
for all $L < L'$. 
Given a classical interaction $I \in \oloc(\sE)$, there is a natural way to attempt construct an effective family of functionals satisfying the RG flow equations.
Indeed, it follows from elementary properties of the homotopy RG flow operator $W(P_{L < L'}, -)$ that {\em if} the functional
\ben
I[L] \;\; ``=" \;\; W(P_{0<L}, I) 
\een
were to be well-defined for each $L >0$, then the RG flow equations would automatically be satisfied for the collection $\{I[L]\}$. 
The problem is that this naive guess is ill-defined due to the distributional nature of the propagator $P_{0<L}$. 
The approach of Costello is to introduce a small parameter $\epsilon > 0$ and to consider the limit of the functionals $W(P_{\epsilon < L}, I)$ as $\epsilon \to 0$. 
For most theories, this $\epsilon \to 0$ limit is ill-defined, but there always exist $\epsilon$-dependent {\em counterterms} $I^{CT}(\epsilon)$ rendering the existence of the $\epsilon \to 0$ limit of $W(P_{\epsilon < L}, I - I^{CT}(\epsilon))$. 

Our main goal in this section amounts to showing that the naive $\epsilon \to 0$ limit exists without the necessity to introduce counterterms. 
This is a salient feature of holomorphic theories on $\CC^d$ that we will take advantage of to characterize anomalies, for instance. 

We will only consider quantizations defined modulo $\hbar^2$.
In this case, the homotopy RG flow takes the explicit form:
\ben
W(P_{\epsilon<L}^V , I) = \sum_{\Gamma} \frac{\hbar^{g(\Gamma)}}{|{\rm Aut}(\Gamma)|} W_\Gamma (P_{\epsilon<L}^V, I) .
\een
The sum is over graphs of genus $\leq 1$ and $W_\Gamma$ is the weight associated to the graph $\Gamma$. 

We can now state the main result of this section.

\begin{prop}\label{lem: hol renorm}
Let $\sE$ be a holomorphic theory on $\CC^d$ with classical interaction $I^{cl}$.  
Then, there exists a one-loop prequantization $\{I[L] \; | \; L > 0\}$ of $I^{cl}$ involving no counterterms. 
That is, the $\epsilon \to 0$ limit of
\ben
W(P_{\epsilon<L} , I) \mod \hbar^2 \in \sO(\sE) [[\hbar]] / \hbar^2
\een
exisits.
Moreover, if $I$ is holomorphically translation invariant we can pick the family $\{I[L]\}$ to be holomorphically translation invariant as well.
\end{prop}

\subsection{Holomorphic gauge fixing}

We follow the mathematically precise approach to renormalization developed by Costello \cite{CosRenorm} which is largely motivated by the original length based regularization due to Wilson. \brian{Give citation}
 
In this formalism, to begin the process of renormalization we must first fix some auxiliary data, that of a {\em gauge fixing operator}.
A gauge fixing operator is a square-zero operator on fields
\ben
Q^{GF} : \sE_V \to \sE_V[-1],
\een
of cohomological degree $-1$ such that $[Q, Q^{GF}]$ is a generalized Laplacian on $\sE$ where $Q$ is the linearized BRST operator. 
For a complete definition see Section 8.2.1 of \cite{CG2}.

For holomorphic theories there is a convenient choice for a gauge fixing operator. 
To construct it we fix the standard flat metric on $\CC^d$. 
Doing this, we let $\dbar^*$ be the adjoint of the operator $\dbar$.
Using the coordinates on $(z_1,\ldots, z_d) \in \CC^d$ we can write this operator as
\ben
\dbar^* = \sum_{i=1}^d \frac{\partial}{\partial (\d \zbar_i)} \frac{\partial}{\partial z_i} .
\een
Equivalently $\frac{\partial}{\partial (\d \zbar_i)}$ is equal to contraction with the anti-holomorphic vector field $\frac{\partial}{\partial \zbar_i}$. 
The operator $\dbar^*$ extends to the complex of fields via the formula
\ben
Q^{GF} = \dbar^* \tensor {\rm id}_V : \Omega^{0,*}(X , V) \to \Omega^{0,*-1}(X, V),
\een

\begin{lem}
The operator $Q^{GF} = \dbar^* \tensor {\rm id}_V$ is a gauge fixing operator for the free theory $(\sE_V, \dbar + Q, \omega_V)$.
\end{lem}
\begin{proof}
Clearly, $Q^{GF}$ is square zero since $(\dbar^*)^2 = 0$.
Since $Q^{hol}$ is a translation invariant holomorphic differential operator we have
\ben
[\dbar + Q^{hol}, Q^{GF}] = [\dbar,\dbar^*] \tensor \id_{V} .
\een
The operator $[\dbar,\dbar^*]$ is simply the Dolbeault Laplacian on $\CC^d$, which 
in coordinates is
\ben
[\dbar,\dbar^*] = -\sum_{i=1}^d \frac{\partial}{\partial \zbar_i}\frac{\partial}{\partial z_i} .
\een
In particular, the operator $[\dbar,\dbar^*] \tensor \id_{V}$ is a generalized Laplacian. 

Finally, we must show that $Q^{GF}$ is (graded) self-adjoint for the shifted symplectic pairing $\omega_V$. 
This follows from the fact about Dolbeualt forms on $\CC^d$.
If $\alpha,\beta \in \Omega^{0,*}_c(\CC^d)$ then
\ben 
\int_{\CC^d} (\dbar^* \alpha) \wedge \beta \wedge \d^d z = \pm \int_{\CC^d} \alpha \wedge (\dbar^* \beta) \wedge \d^d z .
\een
\end{proof}

\subsection{The propagator on $\CC^d$}

The gauge fixing operator determines a generalized Laplaican, which for us is essentially the ordinary Hodge Laplacian on $\CC^d$. 
Our regularization scheme utilizes the heat kernel associated to the Laplacian, for which we recall the explicit form below.

By definition, the scale $L>0$ heat kernel is a symmetric element $K_L^V \in \sE_V(\CC^d) \tensor \sE_V(\CC^d)$ that satisfies
\ben
\omega_V(K_L, \varphi) = e^{-L[Q,Q^{GF}] } \varphi
\een
for any field $\varphi \in \sE_V$.
For a more detailed definition see Section 8.2.3 in \ref{CG2}. 

The tensor square of $\sE_V(\CC^d)$ decomposes as 
\be\label{splitting}
\sE_V(\CC^d) \tensor \sE_V(\CC^d) = \left(\Omega^{0,*}(\CC^d) \tensor \Omega^{0,*}(\CC^d)\right) \tensor (V_0 \tensor V_0) .
\ee
We will decompose the heat kernel accordingly. 

Pick a basis $\{e_i\}$ of $V_0$ and let 
\ben
{\bf C}_{V_0} = \sum_{i,j} \omega_{ij} (e_i \tensor e_j) \in V_0 \tensor V_0
\een
be the quadratic Casimir.
Here, $(\omega_{ij})$ is the inverse matrix to the pairing $(-,-)_{V_0}$. 

The heat kernel splits with respect to the decomposition in Equation \ref{splitting} as
\ben
K_{L}^V (z,w) = K^{an}_L(z,w) \cdot {\bf C}_{V_0} 
\een
where the analytic part is independent of $V$ and equal to
\ben
K_L^{an} (z,w) = \frac{1}{(4 \pi L)^d} e^{-|z-w|^2/ 4L} \prod_{i=1}^d (\d \zbar_i - \d \zbar_j)  \in \Omega^{0,*} (\CC^d) \tensor \Omega^{0,*} (\CC^d) \cong \Omega^{0,*} (\CC^d \times \CC^d) .
\een

The propagator for the holomorphic theory $\sE_V$ is defined by
\ben
P_{\epsilon < L}^V(z,w) = \int_{t=\epsilon}^L \d t (Q^{GF} \tensor 1) K_{L}^V(z,w) .
\een
Since the element ${\bf C}_{V_0}$ is independent of the coordinate on $\CC^d$, the propagator simplifies as $P_{\epsilon < L}^V(z,w) = P_{\epsilon < L}^{an}(z,w) \cdot {\bf C}_{V_0}$ where
\begin{align*}
P_{\epsilon < L}^{an}(z,w) & = \int_{t=\epsilon}^L \d t (\dbar^* \tensor 1) K_{L}^V(z,w) \\
& = \int_{t=\epsilon}^L \d t \frac{1}{(4 \pi t)^d} \sum_{j=1}^d \left(\frac{z_j - w_j}{4 t} \right)  e^{-|z-w|^2 / 4t}  \prod_{i \ne j}^d (\d \zbar_i - \d \zbar_j) .
\end{align*}

\subsection{Trees}

For the genus zero graphs, or trees, we do not have any analytic difficulties to worry about. 
The propagator $P_{\epsilon<L}^V$ is smooth so long as $\epsilon,L > 0$ but when $\epsilon \to 0$ it inherits a singularity along the diagonal $z = w$.
This is what contributes to the divergences in the naive definition of RG flow $W(P_{0<L}, -)$.
But, if $\Gamma$ is a tree the weight $W_\Gamma(P_{0<L}^V, I)$ only involves multiplication of distributions with transverse singular support, so is well-defined.
Thus we have observed the following.

\begin{lem} 
If $\Gamma$ is a tree then $\lim_{\epsilon \to 0} W_{\Gamma}(P_{\epsilon < L}, I)$ exists.
\end{lem}

The only possible divergences in the $\epsilon \to 0$ limit, then, must come from graphs of genus one, which we now direct our attention to.

\subsection{A simplification for one-loop weights}

Every graph of genus one is a wheel with some trees protruding from the external edges of the tree.
Thus, we can write the weight of a genus one graph as a product of weights associated to trees times the weight associated to a wheel.
We have just observed that the weights associated to trees are automatically convergent in the $\epsilon \to 0$ limit, thus it suffices to focus on genus one graphs that are purely wheels with some number of external edges.

The definition of the weight of the wheel involves placing the propagator at each internal edge and the interaction $I$ at each vertex. 
The weights are evaluated by placing compactly supported fields $\varphi \in \sE_{V,c} = \Omega^{0,*}_c(\CC^d, V)$ at each of the external edges.
We will make two simplifications:
\begin{enumerate}
\item the only $\epsilon$ dependence appears in the analytic part of the propagator $P_{\epsilon<L}^{an}$, so we can forget about the combinatorial factor ${\bf C}_{V_0}$ and assume all external edges are labeled by compactly supported Dolbeault forms in $\Omega^{0,*}_c(\CC^d)$;
\item each vertex labeled by $I$ is a sum of interactions of the form
\ben
\int_{\CC^d} D_1(\varphi) \cdots D_k(\varphi) \d^d z
\een
where $D_i$ is a holomorphic differential operator (only involves $\frac{\partial}{\partial z_i}$-derivatives). 
Some of the differential operators will hit the compactly supported Dolbeault forms placed on the external edges of the graph.
The remaining operators will hit the internal edges labeled by the propagators.
Since a holomorphic differential operator preserves the space of compactly supported Dolbeault forms that is independent of $\epsilon$, we replace each input by an arbitrary compactly supported Dolbeault form.
\end{enumerate}

Thus, for the $\epsilon \to 0$ behavior it suffices to look at weights of wheels with arbitrary compactly supported functions as inputs where each of the internal edges are labeled by some translation invariant holomorphic differential operator 
\ben
D = \sum_{n_1,\ldots n_d} \frac{\partial^{n_1}}{\partial z_{1}^{n_1}}\cdots \frac{\partial^{n_d}}{\partial z_{d}^{n_d}}
\een
applied to the propagator $P_{\epsilon<L}^{an}$.
This motivates the following definition. 

\begin{dfn}\label{dfn: analytic weight}
Let $\epsilon , L > 0$. 
In addition, fix the following data.
\begin{enumerate}[(a)]
\item An integer $k \geq 1$ that will be the number of vertices of the graph.
\item For each $\alpha = 1, \ldots, k$ a sequence of integers
\ben
\vec{n}^\alpha = (n_1^\alpha, \ldots, n_d^{\alpha}) .
\een
We denote by $(\vec{n}) = (n_{i}^j)$ the corresponding $d \times k$ matrix of integers. 
\end{enumerate}
The analytic weight associated to the pair $(k, (\vec{n}))$ is the smooth distribution
\ben
W_{\epsilon < L}^{k, (n)} : C_c^\infty((\CC^d)^k) \to \CC,
\een
that sends a smooth compactly supported function $\Phi \in C_c^\infty((\CC^d)^k) = C_c^\infty(\CC^{dk})$ to
\be\label{weight1}
W_{\epsilon < L}^{k, (n)} (\Phi) = \int_{(z^1,\ldots, z^k) \in (\CC^d)^k} \prod_{\alpha=1}^k \d^d z^\alpha \Phi(z^1,\ldots,z^k) \prod_{\alpha = 1}^k \left(\frac{\partial}{\partial z^\alpha}\right)^{\vec{n}^\alpha} P_{\epsilon < L}^{an}(z^\alpha, z^{\alpha+1}) .
\ee
In the above expression, we use the convention that $z^{k+1} = z^1$. 
\end{dfn}

The coordinate on $(\CC^{d})^k$ is given by $\{z_i^\alpha\}$ where $\alpha = 1,\ldots,k$ and $i = 1, \ldots, d$. 
For each $\alpha$, $\{z_1^\alpha, \ldots, z_d^\alpha\}$ is the coordinate for the space $\CC^d$ sitting at the vertex labeled by $\alpha$. 
We have also used the shorthand notation
\ben
\left(\frac{\partial}{\partial z^\alpha}\right)^{\vec{n}^\alpha} = \frac{\partial^{n^\alpha_1}}{\partial z^\alpha_1} \cdots  \frac{\partial^{n^\alpha_d}}{\partial z^\alpha_d}.
\een

We will refer to the collection of data $(k, (\vec{n}))$ in the definition as {\em wheel data}.
The motivation for this is that the weight $W_{\epsilon < L}^{k, (n)}$ is the analytic part of the full weight $W_{\Gamma}(P^V_{\epsilon<L}, I)$ where $\Gamma$ is a wheel with $k$ vertices. 

We have reduced the proof of Proposition \ref{lem: hol renorm} to showing that the $\epsilon \to 0$ limit of the analytic weight $W_{\epsilon < L}^{k, (\vec{n})}(\Phi)$ exists for any choice of wheel data $(k, (\vec{n}))$.
To do this, there are two steps. 
First, we show a vanishing result that says when $k \geq d$ the  weights vanish for purely algebraic reasons. 
The second part is the most technical aspect of the chapter where we show that for $k > d$ the weights have nice asymptotic behavior as a function of $\epsilon$.

\begin{lem} Let $(k, (\vec{n}))$ be a pair of wheel data.
If the number of vertices $k$ satisfies $k \leq d$ then
\ben
W_{\epsilon < L}^{k, (n)}  = 0
\een
as a distribution on $\CC^{dk}$ for any $\epsilon,L > 0$. 
\end{lem}
\begin{proof}
In the integral expression for the weight (\ref{weight1}) there is the following factor involving the product over the edges of the propagators:
\be\label{productprops2}
\prod_{\alpha = 1}^k \left(\frac{\partial}{\partial z^\alpha}\right)^{\vec{n}^\alpha} P_{\epsilon < L}^{an}(z^\alpha, z^{\alpha}) .
\ee
We will show that this expression is identically zero.
To simplify the expression we first make the following change of coordinates on $\CC^{dk}$:
\begin{align}
w^\alpha & = z^{\alpha+1} - z^\alpha \;\;\; , \;\;\; 1\leq \alpha < k \label{coords1}\\
w^k & = z^k \label{coords2} .
\end{align}
Introduce the following operators
\ben
\eta^\alpha = \sum_{i=1}^{d} \wbar_i^\alpha \frac{\partial}{\partial (\d \wbar_i^\alpha)}
\een
acting on differential forms on $\CC^{dk}$.
The operator $\eta^\alpha$ lowers the anti-holomorphic Dolbuealt type by one : $\eta : (p,q) \to (p,q-1)$.
Equivalently, $\eta^\alpha$ is contraction with the anti-holomorphic Euler vector field $\wbar_i^\alpha \partial / \partial \wbar_i^\alpha$.

Once we do this, we see that the expression (\ref{productprops2}) can be written as 
\ben
\left(\left(\sum_{\alpha=1}^{k-1} \eta^\alpha \right) \prod_{i=1}^d \left(\sum_{\alpha = 1}^{k-1} \d \wbar_{i}^\alpha\right) \right) \prod_{\alpha=1}^{k-1}\left( \eta^\alpha \prod_{i=1}^d \d \wbar_i^\alpha\right) .
\een
Note that only the variables $\wbar_i^{\alpha}$ for $i=1,\ldots,d$ and $\alpha = 1,\ldots, k-1$ appear. 
Thus we can consider it as a form on $\CC^{d(k-1)}$.
As such a form it is of Dolbeault type $(0, (d-1) + (k-1)(d-1)) = (0, (d-1)k)$. 
If $k < d$ then clearly $(d-1)k > d(k-1)$ so the form has greater degree than the dimension of the manifold and hence it vanishes. 

The case left to consider is when $k = d$.
In this case, the expression in (\ref{productprops2}) can be written as
\be\label{productprops1}
\left(\left(\sum_{\alpha=1}^{d-1} \eta^\alpha \right) \prod_{i=1}^d \left(\sum_{\alpha = 1}^{d-1} \d \wbar_{i}^\alpha\right) \right) \prod_{\alpha=1}^{d-1}\left( \eta^\alpha \prod_{i=1}^d \d \wbar_i^\alpha\right) .
\ee
Again, since only the variables $\wbar_i^{\alpha}$ for $i=1,\ldots,d$ and $\alpha = 1,\ldots, d-1$ appear, we can view this as a differential form on $\CC^{d(d-1)}$. 
Furthermore, it is a form of type $(0, d(d-1))$. 
For any vector field $X$ on $\CC^{d(d-1)}$ the interior derivative $i_X$ is a graded derivation. 
Suppose $\omega_1,\omega_2$ are two $(0,*)$ forms on $\CC^{d(d-1)}$ such that the sum of their degrees is equal to $d^2$. 
Then, $\omega_1 \iota_X \omega_2$ is a top form for any vector field on $\CC^{d(d-1)}$.
Since $\omega_1 \omega_2 = 0$ for form type reasons, we conclude that $\omega_1 \iota_X \omega_2 = \pm (i_X \omega_1) \omega_2$ with sign depending on the dimension $d$. 
Applied to the vector field $\zbar_i^1\partial / \partial \wbar_i^1$ in (\ref{productprops1}) we see that the expression can be written (up to a sign) as 
\ben
\eta^1 \left(\sum_{\alpha=1}^{d-1} \eta^\alpha \prod_{i=1}^d \left(\sum_{\alpha = 1}^{d-1} \d \wbar_{i}^\alpha\right) \right) \left(\prod_{i=1}^d \d \wbar_i^1\right) \prod_{\alpha=2}^{d-1} \left( \eta^\alpha \prod_{i=1}^d \d \wbar_i^\alpha\right) .
\een
Repeating this, for $\alpha =2,\ldots,k-1$ we can write this expression (up to a sign) as
\ben
\left(\eta_{k-1} \cdots \eta_2 \eta _1 \sum_{\alpha=1}^{k-1} \eta^\alpha \prod_{i=1}^d \left(\sum_{\alpha = 1}^{k-1} \d \wbar_{i}^\alpha\right) \right) \prod_{\alpha=1}^{k-1} \prod_{i=1}^d \d \wbar_i^\alpha 
\een
The expression inside the parentheses is zero since each term in the sum over $\alpha$ involves a term like $\eta^\beta \eta^\beta = 0$. 
This completes the proof for $k=d$. 
\end{proof}

\begin{lem}\label{lem: tech 1}
Let $(k, (\vec{n}))$ be a pair of wheel data such that $k > d$.
Then the $\epsilon \to 0$ limit of the analytic weight
\ben
\lim_{\epsilon \to 0} W_{\epsilon < L}^{k, (n)}
\een
exists as a distribution on $\CC^{dk}$. 
\end{lem}

\begin{proof}

We will bound the absolute value of the weight in Equation (\ref{weight1}) and show that it has a well-defined $\epsilon\to 0$ limit.
First, consider the change of coordinates as in Equations (\ref{coords1}),(\ref{coords2}).
For any compactly supported function $\Phi$ we see that $W_{\epsilon < L}^{k, (n)} (\Phi)$ has the form
\be\label{weight2}
\int_{w^k \in \CC^d} \d^{d} w^k \int_{(w_1,\ldots,w_{k-1}) \in (\CC^d)^{k-1}} \left(\prod_{\alpha=1}^{k-1} \d^{d} w^\alpha\right) \Phi(w^1,\ldots,w^k) \left(\prod_{\alpha=1}^{k-1} \left(\frac{\partial}{\partial w^\alpha}\right)^{\vec{n}^\alpha}P^{an}_{\epsilon < L} (w^\alpha) \right) \sum_{\alpha=1}^{k-1} \left(\frac{\partial}{\partial w^\alpha}\right)^{\vec{n}^k} P^{an}_{\epsilon<L} \left(\sum_{\alpha=1}^{k-1} w^\alpha\right) .
\ee
For $\alpha = 1,\ldots,k-1$ the notation $P^{an}_{\epsilon < L} (w^\alpha)$ makes sense since $P^{an}_{\epsilon<L}(z^\alpha,z^{\alpha+1})$ is only a function of $w^\alpha = z^{\alpha+1}-z^\alpha$.
Similarly $P^{an}_{\epsilon<L}(z^{k+1},z^1)$ is a function of 
\ben
z^k - z^1 = \sum_{\alpha=1}^{k-1} w^\alpha . 
\een
Expanding out the propagators the weight takes the form
\ben
\begin{array}{lll}
& \displaystyle \int_{w^k \in \CC^d} \d^{2d} w^k \int_{(w_1,\ldots,w_{k-1}) \in (\CC^d)^{k-1}} \left(\prod_{\alpha=1}^{k-1} \d^{2d} w^\alpha\right) \Phi(w^1,\ldots,w^k) \int_{(t_1,\ldots,t_k) \in [\epsilon,L]^k} \prod_{\alpha=1}^k \frac{\d t_\alpha}{(4 \pi t_\alpha)^d} \\
& \displaystyle \times \sum_{i_1,\ldots,i_{k-1} =1}^d \epsilon_{i_1\cdots,i_k} \left(\frac{\wbar^1_{i_1}}{4t_1} \frac{(\wbar^1)^{n^1}}{4t^{|n^1|}}\right) \cdots \left(\frac{\wbar^{k-1}_{i_{k-1}}}{4t_{k-1}}\frac{(\wbar^{k-1})^{n^{k-1}}}{4t^{|n^{k-1}|}}\right) \left(\sum_{\alpha=1}^{k-1} \frac{\wbar^\alpha_{i_k}}{4t_k} \cdot \frac{1}{t^{|n^k|}} \left(\sum_{\alpha=1}^{k-1} \wbar^\alpha\right)^{n^k}\right) \\
& \displaystyle \times \exp\left(- \sum_{\alpha=1}^{k-1} \frac{|w^{\alpha}|^2}{4t_\alpha} - \frac{1}{4t_k} \left|\sum_{\alpha=1}^{k-1} w^\alpha \right|^2\right)
\end{array}
\een
The notation used above warrants some explanation. 
Recall, for each $\alpha$ the vector of integers is defined as $n^\alpha = (n^{\alpha}_1,\ldots,n^{\alpha}_d)$. 
We use the notation
\ben
(\wbar^\alpha)^{n^\alpha} = \wbar^{n^\alpha_1}_1 \cdots \wbar^{n^\alpha_d}_d .
\een
Furthermore, $|n^\alpha| = n_1^\alpha + \cdots + n_d^\alpha$. 
Each factor of the form $\frac{\wbar^\alpha_{i_\alpha}}{t_\alpha}$ comes from the application of the operator $\frac{\partial}{\partial z_i}$ in $\dbar^*$ applied to the propagator. 
The factor $\frac{(\wbar^\alpha)^{n^\alpha}}{t^{|n^\alpha|}}$ comes from applying the operator $\left(\frac{\partial}{\partial w}\right)^{n^\alpha}$ to the propagator. 
Note that $\dbar^*$ commutes with any translation invariant holomorphic differential operator, so it doesn't matter which order we do this.

To bound this integral we will recognize each of the factors
\ben
\frac{\wbar^\alpha_{i_\alpha}}{4t_\alpha} \frac{(\wbar^\alpha)^{n^\alpha}}{4t^{|n^\alpha|}}
\een
as coming from the application of a certain holomorphic differential operator to the exponential in the last line.
We will then integrate by parts to obtain a simple Gaussian integral which will give us the necessary bounds in the $t$-variables. 
Let us denote this Gaussian factor by
\ben
E(w,t) := \exp\left(- \sum_{\alpha=1}^{k-1} \frac{|w^{\alpha}|^2}{4t_\alpha} - \frac{1}{4t_k} \left|\sum_{\alpha=1}^{k-1} w^\alpha \right|^2\right)
\een

For each $\alpha,i_\alpha$ introduce the $t=(t_1,\ldots,t_k)$-dependent holomorphic differential operator
\ben
D_{\alpha, i_\alpha}(t) := \left(\frac{\partial}{\partial w^\alpha_{i_\alpha}} - \sum_{\beta = 1}^{k-1} \frac{t_\beta}{t_1+\cdots + t_k} \frac{\partial}{\partial w_{i_\alpha}^{\beta}}\right)
\prod_{j=1}^d \left(\frac{\partial}{\partial w_j^\alpha} - \sum_{\beta =1}^{k-1} \frac{t_\beta}{t_1+\cdots + t_k} \frac{\partial}{\partial w_{j}^\beta}\right)
^{n_j^\alpha} .
\een

The following lemma is an immediate calculation
\begin{lem}\label{lem: diff applied E}
One has
\ben
D_{\alpha,i_\alpha} E(w,t) = \frac{\wbar^\alpha_{i_\alpha}}{4t_\alpha} \frac{(\wbar^\alpha)^{n^\alpha}}{t^{|n^\alpha|}} E(w,t) . 
\een
\end{lem}

Note that all of the $D_{\alpha,i_{\alpha}}$ operators mutually commute. 
Thus, we can integrate by parts iteratively to obtain the following expression for the weight:
\ben
\begin{array}{lll}
& \displaystyle \pm \int_{w^k \in \CC^d} \d^{2d} w^k \int_{(w_1,\ldots,w_{k-1}) \in (\CC^d)^{k-1}}\left(\prod_{\alpha=1}^{k-1} \d^{2d} w^\alpha\right) \int_{(t_1,\ldots,t_k) \in [\epsilon,L]^k} \prod_{\alpha=1}^k \frac{\d t_\alpha}{(4 \pi t_\alpha)^d}  \\ 
& \displaystyle \times\left( \sum_{i_1,\ldots, i_k} \epsilon_{i_1\cdots,i_d} D_{1, i_1} \cdots D_{k-1,i_{k-1}} \sum_{\alpha=1}^{k-1} D_{\alpha, i_k} \Phi(w^1,\ldots,w^k) \right) \times \exp\left(- \sum_{\alpha=1}^{k-1} \frac{|w^{\alpha}|^2}{t_\alpha} - \frac{1}{t_k} \left|\sum_{\alpha=1}^{k-1} w^\alpha \right|^2\right) .
\end{array}
\een
%\brian{all the differential operators $D_{\alpha, i_\alpha}$ are uniformly bounded in $t$. To make these precise I should find what the uniform bound is.}

Thus, the absolute value of the weight is bounded by 
\be\label{weight bound1}
|W_{\epsilon < L}^{k, (n)}(\Phi)| \leq C \int_{w^k \in \CC^d} \d^{2d} w^k \int_{(w^1,\ldots,w^{k-1}} \prod_{\alpha=1}^{k-1} \d^{2d} w^\alpha \Psi(w^1,\ldots,w^{k-1},w^k) \int_{(t_1,\ldots,t_k) \in [\epsilon,L]^k} \d t_1 \ldots \d t_k \frac{1}{(4\pi)^{dk}} \frac{1}{t^d_1\cdots t^d_k} \times E(w,t)
\ee
where $\Psi$ is some compactly supported functnio on $\CC^{dk}$ that is independent of $t$. 

To compute the right hand side we will perform a Gaussian integration with respect to the variables $(w^1,\ldots,w^{k-1})$. 
To this end, notice that the exponential can be written as
\ben
E(w,t) = \exp\left(-\frac{1}{4} M_{\alpha\beta} (w^\alpha, w^\beta)\right)
\een
where $(M_{\alpha\beta})$ is the $(k-1)\times (k-1)$ matrix given by
\ben
\begin{pmatrix}
a_1 & b & b & \cdots & b \\
b & a_2 & b & \cdots & b \\
b & b & a_3 & \cdots & b \\
\vdots & \vdots & \vdots &  \ddots & \vdots \\
b & b & b & \cdots & a_{k-1}
\end{pmatrix} 
\een
where $a_\alpha = t_\alpha^{-1} + t_k^{-1}$ and $b = t_k^{-1}$.
The pairing $(w^{\alpha}, w^{\beta})$ is the usual Hermitian pairing on $\CC^d$, $(w^{\alpha}, w^{\beta}) = \sum_i w^{\alpha}_i \wbar^\beta_i$.
After some straightforward linear algebra we find that 
\ben
\det(M_{\alpha\beta})^{-1} = \frac{t_1\cdots t_k}{t_1+\cdots+t_k} .
\een 
We now perform a Wick expansion for the Gaussian integral in the variables $(w^1,\ldots,w^{k-1})$.
For a reference similar to the notation used here see the Appendix of our work in \cite{EWY}.
The inequality in (\ref{weight bound1}) becomes
\begin{align}\label{weight bound2}
|W_{\epsilon < L}^{k, (n)}(\Phi)| & \leq C' \int_{w^k \in \CC^d} \d^{2d} w^k \Psi(0, \ldots, 0, w^k) \int_{(t_1,\ldots,t_k) \in [\epsilon,L]^k} \d t_1 \ldots \d t_k \frac{1}{(4\pi)^{dk}} \frac{1}{(t_1\cdots t_k)^d}\left(\frac{t_1\cdots t_k}{t_1+\cdots+t_k}\right)^d + O(\epsilon) \\ & = C' \int_{w^k \in \CC^d} \d^{2d} w^k \Psi(0, \ldots, 0, w^k) \int_{(t_1,\ldots,t_k) \in [\epsilon,L]^k} \d t_1 \ldots \d t_k \frac{1}{(4\pi)^{dk}} \frac{1}{(t_1+\cdots+t_k)^d} + O(\epsilon) .
\end{align}
The first term in the Wick expansion is written out explicitly. 
The $O(\epsilon)$ refers to higher terms in the Wick expansion, which one can show all have order $\epsilon$, so disappear in the $\epsilon \to 0$ limit.
The expression $\Psi(0, \ldots, 0, w^k)$ means that we have evaluate the function $\Psi(w^1,\ldots, w^k)$ at $w^1=\ldots=w^{k-1} =0$ leaving it as a function only of $w^k$. 
In the original coordinates this is equivalent to setting $z^1=\cdots=z^{k-1} = z^k$.

Our goal is to show that $\epsilon \to 0$ limit of the right-hand side exists. 
The only $\epsilon$ dependence on the right hand side of (\ref{weight bound2}) is in the integral over the regulation parameters $t_1,\ldots, t_k$. 
Thus, it suffices to show that the $\epsilon \to 0$ limit of 
\ben
\int_{(t_1,\ldots,t_k) \in [\epsilon,L]^k} \frac{\d t_1 \ldots \d t_k}{(t_1+\cdots+t_k)^d}
\een
exists.
By the AM/GM inequality we have $(t_1+\cdots+t_k)^d \geq (t_1 \cdots t_d)^{d/k}$. 
So, the integral is bounded by
\ben
\int_{(t_1,\ldots,t_k) \in [\epsilon,L]^k}\frac{\d t_1 \ldots \d t_k}{(t_1+\cdots+t_k)^d} \leq \int_{(t_1,\ldots,t_k) \in [\epsilon,L]^k}\frac{\d t_1 \ldots \d t_k }{(t_1 \cdots t_k)^{d/k}} = \frac{1}{(1-d/k)^k} \left(\epsilon^{1-d/k} - L^{1-d/k}\right)^k .
\een
By assumption, $d < k$, so the right hand side has a well-defined $\epsilon \to 0$ limit. 
This concludes the proof.

%Now, since $t_\alpha / \sum_\beta t_\beta < 1$ for each $\alpha$ we have the following bound for the operators $D_{\alpha, i_\alpha}$:
%\bestar
%\left|D_{\alpha,i_{\alpha}} \Phi\right| & \leq \left(\left|\frac{\partial}{\partial w^\alpha_{i_\alpha}} \Phi\right| +  \sum_{\beta = 1}^{k-1}\frac{\partial}{\partial w_{i_\beta}^{\beta}}\right)
%\prod_{j=1}^d \left(\frac{\partial}{\partial w_j^\alpha} - \frac{1}{k} \sum_{\beta =1}^{k-1} \frac{\partial}{\partial w_{j}^\beta}\right)
%^{n_j^\alpha} \right| 
\end{proof}

\section{Applications}

\subsection{One-loop renormalization for topological theories}

\subsection{Chiral anomalies in arbitrary dimensions}


\end{document}

